\chapter{Aktueller Stand der Performance Analyse in der \acl{CE}}
Die \ac{CE} nutzt bereits verschiedene Wege um die Performance und das
Verhalten der HANA Datenbank zu analysieren. 
\section{Google Benchmark}
\label{sec:google_benchmark}

Google Benchmark ist ein Open Source Benchmarking Tool von Google,
welches es einem ermöglicht einzelne Funktionen zu benchmarken. 

Dazu kann man ähnlich zu den meisten Test Frameworks drei verschiedene Funktionen definieren.

Die Hauptfunktion definiert was genau im Benchmark untersucht werden
soll. Diese legt die für die Messung relevante Logik fest.

Die Setup Funktion wird einmal vor jeder Ausführung der Hauptfunktion
aufgerufen. In dieser werden die Voraussetzungen für die Ausführung der Hauptfunktion
geschaffen. Man könnte sie beispielsweise nutzen, um Testdaten für den Benchmark
zu generieren oder zu laden, da sie nicht bei den Messungen beachtet wird.

Die Teardown Funktion wird nach jeder Ausführung der Hauptfunktion aufgerufen.
Diese beeinflusst, wie die Setup Funktion, die Messung nicht.

Des Weiteren bietet Google Benchmark die Option, einen Benchmark mehrmals mit
unterschiedlichen Parametern durchzuführen. Um beispielsweise die Auswirkung
der Größe des Testdatensatzes auf das Ergebnis zu beobachten.

\autoref{fig:google_benchmark_ausgabe} zeigt eine Beispielhafte Google
Benchmark Ausgabe. Benchmark ist dabei der Name des Benchmarks und hinter dem
Schrägstrich die Größe des Inputs. Time ist die durchschnittliche Dauer einer
Ausführung der Hauptfunktion über alle Iterationen hinweg. CPU ist die
durchschnittliche CPU Zeit einer Ausführung der Hauptfunktion über alle
Iterationen hinweg. Iterations ist die Anzahl der durchgeführten Iterationen,
bis ein stabiler Durchschnittswert erreicht wurde. \autocite[Vgl.][]{GoogleBenchmark}

\begin{figure}[h]
    \begin{center}
        \begin{verbatim}
---------------------------------------------------------------
Benchmark                          Time         CPU Iterations
---------------------------------------------------------------
BM_RemoveDetachedNodes/2           4 us        4 us     189823 
BM_RemoveDetachedNodes/8           9 us        9 us      78165 
BM_RemoveDetachedNodes/64         58 us       58 us      12106 
BM_RemoveDetachedNodes/512       469 us      468 us       1515 
BM_RemoveDetachedNodes/4096     4237 us     4234 us        166 
BM_RemoveDetachedNodes/8192    10026 us    10019 us         60 
        \end{verbatim}
    \end{center}
    \caption{Google Benchmark Ausgabe}\label{fig:google_benchmark_ausgabe}
\end{figure}

Google Benchmark wird in der \ac{CE} meistens genutzt, um das Verhalten der
Laufzeit bestimmter Funktionen in künstlich generierten Testfällen zu
vergleichen. Hierbei wird jedoch keine HANA Instanz benötigt, da
keine Operationen auf einer Datenbank ausgeführt werden. Es werden folglich
auch keine Testdatensätze für die Datenbank benötigt, sondern nur Daten, auf
welchen man die zu analysierende Methode aufrufen kann. Solche Performance
Tests sind also im Vergleich zu Tests, welche auf eine HANA Instanz mit
Testdatensätzen zugreifen, relativ einfach. Da sie sich weniger Voraussetzungen
haben, und nur auf einen kleinen Teil der Logik fokussiert sind.

\section{HANA Profiler}
\label{sec:hana_profiler}
