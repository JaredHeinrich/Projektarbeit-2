\chapter{Grundlagen}

\section{\acl{CE}}

\section{Ausführung von Datenbankabfragen}
\label{sec:ausfuehrung_db_queries}
Datenbankabfragen werden in der \todo{Wo genau werden die Abfragen ausgeführt?
Was gehört alles zur CalcEngine?} \ac{CE} in drei Schritten ausgeführt. Zuerst
wird die Calculation View instanziiert. Anschließend wird dieses mithilfe
verschiedener Methoden optimiert. Dieses optimierte Modell wird dann auf der
Datenbank ausgeführt.

% section Ausführung von Datenbankabfragen (end)

\section{Verschiedene Arten von Performance Analyse}
\label{sec:arten_performance_analyse}
\subsection*{Benchmarking}
\subsection*{Profiling}

% section Verschiedene Arten von Performance Analyse (end)

\section{Notwendige Grundlagen aus der Graphentheorie}
\label{sec:grundlagen_graphentheorie}

\autoref{def:gerichteter_graph} gibt eine Definition für einen gerichteten
Graphen nach \autocite[Vgl.][S.220]{AlgorithmenUndDatenstrukturen}.
\begin{definition}
    Ein gerichteter Graph ist ein Tupel $G=(V,E)$. $V$ heißt Menge der Knoten.
    $E$ heißt Menge der gerichteten Kanten. Es gilt $V\ne\emptyset$ und $E \subset V \times
    V \setminus \{(v,v) | v \in V\}$. Zwischen zwei Knoten $u,v \in V$ gibt es
    eine Kante mit dem Anfangspunkt $u$ und dem Endpunkt $v$, wenn $(u,v) \in E$.
    \label{def:gerichteter_graph}
\end{definition}

\autoref{def:pfad} gibt eine Definition für einen Pfad in einem Graphen nach
\autocite[Vgl.][S.221f]{AlgorithmenUndDatenstrukturen}.
\begin{definition}
    Ein Pfad $P$ in $G$  ist eine Folge von Knoten $v_0, \dots ,v_n$. Dabei
    muss $\forall i \in \{0, \dots, n-1 \} : (v_i,v_{i+1}) \in E$ gelten. $v_0$
    heißt Anfangspunkt von $P$, $v_n$ heißt Endpunkt von $P$ und $n$ heißt
    Länge von $P$.
    \label{def:pfad}
\end{definition}

\autoref{def:zyklen} gibt nach
\autocite[Vgl.][S.222]{AlgorithmenUndDatenstrukturen} eine Definition für
Zyklen in einem gerichteten Graphen und definiert den Begriff des azyklischen
Graphen.
\begin{definition}
    Ein Pfad heißt geschlossen, wenn $v_0 = v_n$. Ein
    geschlossener Pfad heißt einfach, wenn $\forall i,j \in \{0,\dots,n-1\}, i
    \ne j:v_i \ne v_j$ gilt. In einem gerichteten Graphen heißt ein einfach
    geschlossener Pfad mit $n\geq 2$ auch Zyklus. Ein Graph heißt azyklisch,
    wenn er keine Zyklen besitzt.
    \label{def:zyklen}
\end{definition}

\autoref{def:gewichteter_graph} definiert den Begriff des gewichteten Graphen
nach \autocite[Vgl.][S.253]{AlgorithmenUndDatenstrukturen}.
\begin{definition}
    $G$ heißt gewichtet, wenn es eine Abbildung $g:E\rightarrow \mathbb{R}$
    gibt, welche jeder Kante ein Gewicht zuordnet. Für $e\in E$ heißt $g(e)$
    Gewicht von $e$.
    \label{def:gewichteter_graph}
\end{definition}

\autoref{def:child_parent} definiert die Begriffe Kindknoten, Elternknoten und
Subknoten.
\begin{definition}
    Für zwei Knoten $c$ und $p$ gilt, $c$ ist Kindknoten von $p$, wenn $(p,c)\in V$.
    Umgekehrt gilt, wenn $(p,c)\in V$, $p$ ist Elternknoten von $c$.
    $c$ ist Subknoten von $p$, wenn $\exists P: (v_0=p) \land (v_n=c)$.
    \label{def:child_parent}
\end{definition}
% section Notwendige Grundlagen aus der Graphentheorie (end)

\section{Statistische Grundlagen}
\label{sec:statistische_grundlagen}
% Metrisches Merkmal
In \autoref{def:metrisches_merkmal} wird das metrische Merkmal nach
\autocite[Vgl.][S.24]{Statistik} definiert.
\begin{definition}
    Bei einem metrisch skaliertem Merkmale werden einzelne Ausprägungen anhand
    einer Zahlenskala bewertet, häufig in Verbindung mit einer Maßeinheit. Sie
    lassen sich anhand der Größe ordnen und vergleichen. Außerdem kann man die
    Abstände zwischen verschiedenen Ausprägungen messen und interpretieren.
    \label{def:metrisches_merkmal}
\end{definition}

% Arithmetisches Mittel
\autoref{def:durchschnitt} definiert die, im Kontext dieser Arbeit gleichbedeutenden, Begriffe
Durchschnitt, Mittelwert und Arithmetisches Mittel nach \autocite[Vgl.][S.52f]{Statistik}.
\begin{definition}
    Das arithmetische Mittel von $n$ metrisch skalierten Beobachtungswerten
    $x_1,\dots,x_n$, genannt $\bar{x}$. 

    Es gilt: $\bar{x} = \frac{1}{n} \displaystyle\sum^{n}_{i=1}x_i$

    \label{def:durchschnitt}
\end{definition}

% Varianz
\autoref{def:varianz} beschreibt den Begriff der Varianz nach
\autocite[Vgl.][S.399]{CorrelationBetweenRelatives}.
\begin{definition}
    Die empirische Varianz von $n$ metrisch skalierten Beobachtungswerten $x_1,
    \dots, x_n$, genannt $\tilde{s}^2$, ist der Durchschnitt der quadratischen
    Abweichung von $x_i$ von $\bar{x}$.

    Es gilt: $\tilde{s}^2 = \frac{1}{n} \displaystyle\sum^{n}_{i=1}(x_i - \bar{x})^2 $
    \label{def:varianz}
\end{definition}
% ?Standartabweichung?
% ?Regression?
% section Statistische Grundlagen (end)
