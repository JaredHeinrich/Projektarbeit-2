\chapter{Grundlagen}

\section{\acl{CE}}

\section{Modelle und Datenbankabfragen}
\label{sec:modells_db_queries}
In der \ac{CE} werden Datenbankabfragen durch Modelle dargestellt.
Für jede Abfrage wird zuerst das dazugehörige Modell instanziiert.
Anschließend wird dieses Modell optimiert, um die Ausführungsdauer der Abfrage zu minimieren. 
Diese optimierte Abfrage, wird dann auf der Datenbank ausgeführt.
Allgemein kann man ein Modell als azyklischen gerichteten Graphen nach
\autoref{def:gerichteter_graph} und \autoref{def:zyklen} beschreiben.
Des Weiteren ist zu beachten, dass dieser Graph nur eine Quelle nach
\autoref{def:quelle_senke} hat, welche auch als Abfrageknoten bezeichnet wird.
Diese Definitionen reichen jedoch nicht aus, da zwischen verschiedenen
Arten von Knoten unterschieden wird, welche jeweils verschiedene Informationen
beinhalten. Die wichtigsten dieser Knoten Typen sind
\foreignlanguage{english}{Projection}, \foreignlanguage{english}{Aggregation},
\foreignlanguage{english}{Join}, \foreignlanguage{english}{Union} und
\foreignlanguage{english}{Table}.

%\subsubsection*{Tabellen- und Operationsknoten}
Allgemein werden die Knoten in der \ac{CE} in zwei Gruppen unterschieden.
Die Datenquellenknoten, zu welchen \zB der
\foreignlanguage{english}{Table} Knotentyp gehört. Diese werden im Folgenden
auch Tabellenknoten genannt, da dieser der einzig relevante der Gruppe ist.
Die Operationsknoten, zu welchen die anderen genannten Knotentypen gehören.
Ein \foreignlanguage{english}{Table}-Knoten spiegelt als Datenquellenknoten
eine Tabelle einer Datenbank wider. Ein Operationsknoten bildet hingegen eine
Datenbankoperation ab.

%\subsubsection*{Eingangs- und Ausgangsknoten}
Zwar könnte man das Modell wie in \autoref{def:gerichteter_graph} beschreiben
als eine Menge von Knoten und Kanten darstellen, tatsächlich wird es jedoch nur
als eine Menge von Knoten gespeichert, wobei in jedem Knoten festgehalten ist,
wer seine Kind- und Elternknoten sind. Die Kindknoten werden dabei als
Eingangsknoten und die Elternknoten als Ausgangsknoten bezeichnet.
Jeder Knoten kann beliebig viele Ausgangsknoten haben, wobei es wie
bereits beschreiben nur einen Knoten mit $0$ Ausgangsknoten gibt.
Die Anzahl der Eingangsknoten unterscheidet sich dabei je nach Knotentyp.
\foreignlanguage{english}{Projection}- und
\foreignlanguage{english}{Aggregation}-Knoten haben genau einen, ein
\foreignlanguage{english}{Join}-Knoten hat genau zwei, ein
\foreignlanguage{english}{Union}-Knoten hat zwei oder mehr und ein
\foreignlanguage{english}{Table}-Knoten hat keinen Eingangsknoten.

\autoref{fig:bsp_modell} zeigt ein einfaches Beispielmodell, welches zur
Veranschaulichung dient. Das Modell besteht aus fünf Knoten, dem Abfrageknoten,
zwei Operationsknoten und zwei Tabellenknoten. In dem Knoten steht der Name des
Knotens und neben ihm steht der Knotentyp. Bis auf den Abfrageknoten haben in
diesem Beispiel alle Knoten eine Zahl als Namen. Der Abfrageknoten sowie alle
Tabellenknoten sind zusätzlich farblich markiert. Knoten 1 ist ein
\foreignlanguage{english}{Join}-Knoten. Er hat zwei Eingangsknoten,
den \foreignlanguage{english}{Table}-Knoten 2 und den
\foreignlanguage{english}{Projection}-Knoten 3. Der einzige Ausgangsknoten ist in diesem Fall der
Abfrageknoten, dieser ist hier vom Typ \foreignlanguage{english}{Aggregation}.
Der Abfrageknoten könnte jedoch ein beliebiger Operationsknoten sein.

\begin{figure}
    \begin{center}
        \begin{tikzpicture}[
    node distance = 1cm,
    request/.style = {rectangle, draw, blue, minimum width=2cm, minimum height=0.8cm},
    op_node/.style = {circle, draw, minimum size=1cm},
    table_node/.style = {circle, draw, magenta, minimum size=1cm},
    arrow/.style = {->, >=stealth}
]

% Nodes
\node[request] (0) at (0,3) {Abfrage};
\node[right=0.2cm of 0] {Aggregation};

\node[op_node] (1) at (0,1) {1};
\node[below=0.2cm of 1] {Join};

\node[table_node] (2) at (-2,-1) {2};
\node[below=0.2cm of 2] {Table};

\node[op_node] (3) at (2,-1) {3};
\node[right=0.2cm of 3] {Projection};

\node[table_node] (4) at (2,-3) {4};
\node[below=0.2cm of 4] {Table};

% Arrows
\draw[arrow] (0) -- (1);
\draw[arrow] (1) to[out=210,in=90] (2);
\draw[arrow] (1) to[out=-30,in=90] (3);
\draw[arrow] (3) -- (4);

\end{tikzpicture}

    \end{center}
    \caption{Beispiel Modell}\label{fig:bsp_modell}
\end{figure}

Jeder Knoten beinhaltet eine Liste an Sichtattributen, diese legt fest, welche
Sichtattribute seiner Eingangsknoten dieser Knoten an seine Ausgangsknoten
weitergibt. Bei einem Tabellenknoten sind die Sichtattribute gleichbedeutend
mit den Spalten der Tabelle. Die Sichtattribute des Abfrageknotens sind die
Attribute, welche im Ergebnis der Abfrage angezeigt werden.

% section Modelle und Datenbankabfragen (end)

\section{Verschiedene Arten von Performance Analyse}
\label{sec:arten_performance_analyse}
\subsubsection*{Benchmarking}
\subsubsection*{Profiling}

% section Verschiedene Arten von Performance Analyse (end)

\section{Notwendige Grundlagen aus der Graphentheorie}
\label{sec:grundlagen_graphentheorie}

\autoref{def:gerichteter_graph} gibt eine Definition für einen gerichteten
Graphen \autocite[Vgl.][S.220]{AlgorithmenUndDatenstrukturen}.
\begin{definition}
    Ein gerichteter Graph ist ein Tupel $G=(V,E)$. $V$ heißt Menge der Knoten.
    $E$ heißt Menge der gerichteten Kanten. Es gilt $V\ne\emptyset$ und $E \subset V \times
    V \setminus \{(v,v) | v \in V\}$. Zwischen zwei Knoten $u,v \in V$ gibt es
    eine Kante mit dem Anfangspunkt $u$ und dem Endpunkt $v$, wenn $(u,v) \in E$.
    \label{def:gerichteter_graph}
\end{definition}

\autoref{def:pfad} gibt eine Definition für einen Pfad in einem Graphen
\autocite[Vgl.][S.221f]{AlgorithmenUndDatenstrukturen}.
\begin{definition}
    Ein Pfad $P$ in $G$  ist eine Folge von Knoten $v_0, \dots ,v_n$. Dabei
    muss $\forall i \in \{0, \dots, n-1 \} : (v_i,v_{i+1}) \in E$ gelten. $v_0$
    heißt Anfangspunkt von $P$, $v_n$ heißt Endpunkt von $P$ und $n$ heißt
    Länge von $P$.
    \label{def:pfad}
\end{definition}

\autoref{def:zyklen} gibt  eine Definition für Zyklen in einem gerichteten
Graphen und definiert den Begriff des azyklischen Graphen
\autocite[Vgl.][S.222]{AlgorithmenUndDatenstrukturen}.
\begin{definition}
    Ein Pfad heißt geschlossen, wenn $v_0 = v_n$. Ein
    geschlossener Pfad heißt einfach, wenn $\forall i,j \in \{0,\dots,n-1\}, i
    \ne j:v_i \ne v_j$ gilt. In einem gerichteten Graphen heißt ein einfach
    geschlossener Pfad mit $n\geq 2$ auch Zyklus. Ein Graph heißt azyklisch,
    wenn er keine Zyklen besitzt.
    \label{def:zyklen}
\end{definition}

\autoref{def:gewichteter_graph} definiert den Begriff des gewichteten Graphen
\autocite[Vgl.][S.253]{AlgorithmenUndDatenstrukturen}.
\begin{definition}
    $G$ heißt gewichtet, wenn es eine Abbildung $g:E\rightarrow \mathbb{R}$
    gibt, welche jeder Kante ein Gewicht zuordnet. Für $e\in E$ heißt $g(e)$
    Gewicht von $e$.
    \label{def:gewichteter_graph}
\end{definition}

\autoref{def:quelle_senke} definiert die Begriffe Quelle und Senke
\autocite[Vgl.][S.306]{AlgorithmenUndDatenstrukturen}.
\begin{definition}
    Ein Knoten $q \in V$ heißt Quelle, wenn $ \neg \exists p \in V: (p,q) \in E
    $, $q$ also nach \autoref{def:kind_eltern} keine Elternknoten hat.
    $s \in V$ heißt Senke, wenn $ \neg \exists p \in V: (s,p) \in E $, $s$ also
    nach \autoref{def:kind_eltern} keine Kindknoten hat.
    \label{def:quelle_senke}
\end{definition}

\autoref{def:kind_eltern} definiert die Begriffe Kindknoten, Elternknoten und
Subknoten.
\begin{definition}
    Für zwei Knoten $c$ und $p$ gilt, $c$ ist Kindknoten von $p$, wenn $(p,c)\in E$.
    Umgekehrt gilt, wenn $(p,c)\in E$, $p$ ist Elternknoten von $c$.
    $c$ ist Subknoten von $p$, wenn $\exists P: (v_0=p) \land (v_n=c)$, es also
    einen Pfad von $p$ nach $c$ gibt.
    \label{def:kind_eltern}
\end{definition}
% section Notwendige Grundlagen aus der Graphentheorie (end)

\section{Statistische Grundlagen}
\label{sec:statistische_grundlagen}
% Metrisches Merkmal
In \autoref{def:metrisches_merkmal} wird der Begriff metrisches Merkmal
definiert  \autocite[Vgl.][S.24]{Statistik}.
\begin{definition}
    Bei einem metrisch skaliertem Merkmale werden einzelne Ausprägungen anhand
    einer Zahlenskala bewertet, häufig in Verbindung mit einer Maßeinheit. Sie
    lassen sich anhand der Größe ordnen und vergleichen. Außerdem kann man die
    Abstände zwischen verschiedenen Ausprägungen messen und interpretieren.
    \label{def:metrisches_merkmal}
\end{definition}

% Arithmetisches Mittel
\autoref{def:durchschnitt} definiert die, im Kontext dieser Arbeit gleichbedeutenden, Begriffe
Durchschnitt, Mittelwert und Arithmetisches Mittel \autocite[Vgl.][S.52f]{Statistik}.
\begin{definition}
    Das arithmetische Mittel von $n$ metrisch skalierten Beobachtungswerten
    $x_1,\dots,x_n$, genannt $\bar{x}$. 

    Es gilt: $\bar{x} = \frac{1}{n} \displaystyle\sum^{n}_{i=1}x_i$

    \label{def:durchschnitt}
\end{definition}

% Varianz
\autoref{def:varianz} beschreibt den Begriff der Varianz
\autocite[Vgl.][S.399]{CorrelationBetweenRelatives}.
\begin{definition}
    Die empirische Varianz von $n$ metrisch skalierten Beobachtungswerten $x_1,
    \dots, x_n$, genannt $\tilde{s}^2$, ist der Durchschnitt der quadratischen
    Abweichung von $x_i$ von $\bar{x}$.

    Es gilt: $\tilde{s}^2 = \frac{1}{n} \displaystyle\sum^{n}_{i=1}(x_i - \bar{x})^2 $
    \label{def:varianz}
\end{definition}

% Standartabweichung
\autoref{def:standardabweichung} beschreibt den Begriff der Standardabweichung
\autocite[Vgl.][S.80]{ContributionsToTheMathematicalTheory}.
\begin{definition}
    Die empirische Standardabweichung von $n$ metrisch skalierten Beobachtungswerten $x_1,
    \dots, x_n$, genannt $\tilde{s}$, ist die Quadratwurzel der Varianz.

    Es gilt: $\tilde{s} = \sqrt{\tilde{s}^2} = \sqrt{\frac{1}{n}
    \displaystyle\sum^{n}_{i=1}(x_i - \bar{x})^2} $
    \label{def:standardabweichung}
\end{definition}

% ?Regression?
% section Statistische Grundlagen (end)
