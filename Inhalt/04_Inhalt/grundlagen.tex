\chapter{Grundlagen}

\section{\acl{CE}}
%HANA?

\section{Modelle und Datenbankabfragen}
\label{sec:modells_db_queries}
Um Daten einer HANA Datenbank zu analysieren, können sogenannte
Kalkulationssichten genutzt werden. Da nur diese Kalkulationssichten in der
Arbeit weiter betrachtet werden, werden sie im Folgenden auch als
Datenbankabfragen bezeichnet. \autocite[Vgl.][]{SapHanaCreateCalcViews}

In der \ac{CE} werden diese Datenbankabfragen durch Modelle dargestellt. Für
jede Abfrage wird zuerst das dazugehörige Modell instanziiert. Anschließend
wird dieses Modell optimiert, um die Ausführungsdauer der Abfrage zu
minimieren.  Diese optimierte Abfrage, wird dann auf der Datenbank ausgeführt.
Allgemein kann man ein Modell als azyklischen gerichteten Graphen nach
\autoref{def:gerichteter_graph} und \autoref{def:zyklen} beschreiben. Des
Weiteren ist zu beachten, dass dieser Graph nur eine Quelle nach
\autoref{def:quelle_senke} hat, welche auch als Abfrageknoten bezeichnet wird.
Diese Definitionen reichen jedoch nicht aus, da zwischen verschiedenen Arten
von Knoten unterschieden wird, welche jeweils verschiedene Informationen
beinhalten.  Allgemein werden die Knoten in der \ac{CE} in zwei Gruppen
unterschieden, Datenquellen und Sichtknoten. Es gibt verschiedene Datenquellen,
zur Vereinfachung werden in dieser jedoch Arbeit nur
\foreignlanguage{english}{Table}-Knoten betrachtet. Deshalb werden Datenquellen
im Folgenden auch Tabellenknoten genannt.
%Ein \foreignlanguage{english}{Table}-Knoten spiegelt eine Tabelle einer Datenbank wider.
Die andere Gruppe sind die Sichtknoten. Zu diesen gehören \zB
\foreignlanguage{english}{Projection}, \foreignlanguage{english}{Aggregation},
\foreignlanguage{english}{Join} und \foreignlanguage{english}{Union}. Auch hier
gibt es zwar noch Weitere, diese Arbeit beschränkt sich jedoch auf diese.
Diese Modelle könnte man zwar wie in \autoref{def:gerichteter_graph}
beschreiben als eine Menge von Knoten und Kanten darstellen, gespeichert wird
jedoch eine Liste an Knoten, von welchen jedem seine Kind- und Elternknoten
zugeordnet sind. Die Kindknoten werden dabei als Eingangsknoten und die
Elternknoten als Ausgangsknoten bezeichnet. Jeder Knoten kann beliebig viele
Ausgangsknoten haben, wobei es wie bereits beschreiben nur einen Knoten mit $0$
Ausgangsknoten gibt. Die Anzahl der Eingangsknoten unterscheidet sich dabei je
nach Knotentyp.
\foreignlanguage{english}{Projection}- und
\foreignlanguage{english}{Aggregation}-Knoten haben genau einen, \todo{Was genau ist Aggregation}
\foreignlanguage{english}{Join}- \todo{Join nur 2}und \foreignlanguage{english}{Union}-Knoten
haben zwei oder mehr und ein \foreignlanguage{english}{Table}-Knoten hat keinen
Eingangsknoten. Bei den Eingangsknoten kann es sich um Tabellenknoten
sowie auch Sichtknoten handeln. \autocite[Vgl.][/Create Calculation
Views/Supported View Nodes for Modeling Calculation
Views]{SapHanaCreateCalcViews}

\autoref{fig:bsp_modell} zeigt ein einfaches Beispielmodell, welches zur
Veranschaulichung dient. Das Modell besteht aus fünf Knoten, dem Abfrageknoten,
zwei Sichtknoten und zwei Tabellenknoten. In dem Knoten steht der Name des
Knotens und neben ihm steht der Knotentyp. Bis auf den Abfrageknoten haben in
diesem Beispiel alle Knoten eine Zahl als Namen. Der Abfrageknoten sowie alle
Tabellenknoten sind zusätzlich farblich markiert. Knoten 1 ist ein
\foreignlanguage{english}{Join}-Knoten. Er hat zwei Eingangsknoten,
den \foreignlanguage{english}{Table}-Knoten 2 und den
\foreignlanguage{english}{Projection}-Knoten 3. Der einzige Ausgangsknoten ist in diesem Fall der
Abfrageknoten, dieser ist hier vom Typ \foreignlanguage{english}{Aggregation}.
Der Abfrageknoten könnte jedoch alternativ auch vom Typ \foreignlanguage{english}{Projection} sein. \todo{Unsicher??}

\begin{figure}
    \begin{center}
        \begin{tikzpicture}[
    node distance = 1cm,
    request/.style = {rectangle, draw, blue, minimum width=2cm, minimum height=0.8cm},
    op_node/.style = {circle, draw, minimum size=1cm},
    table_node/.style = {circle, draw, magenta, minimum size=1cm},
    arrow/.style = {->, >=stealth}
]

% Nodes
\node[request] (0) at (0,3) {Abfrage};
\node[right=0.2cm of 0] {Aggregation};

\node[op_node] (1) at (0,1) {1};
\node[below=0.2cm of 1] {Join};

\node[table_node] (2) at (-2,-1) {2};
\node[below=0.2cm of 2] {Table};

\node[op_node] (3) at (2,-1) {3};
\node[right=0.2cm of 3] {Projection};

\node[table_node] (4) at (2,-3) {4};
\node[below=0.2cm of 4] {Table};

% Arrows
\draw[arrow] (0) -- (1);
\draw[arrow] (1) to[out=210,in=90] (2);
\draw[arrow] (1) to[out=-30,in=90] (3);
\draw[arrow] (3) -- (4);

\end{tikzpicture}

    \end{center}
    \caption{Beispiel Modell}\label{fig:bsp_modell}
\end{figure}

Zusätzlich zu den Eingangs- und Ausgangsknoten werden in jedem Knoten
noch weitere Informationen gespeichert. Jeder Knoten beinhaltet
eine Liste an Sichtattributen, diese legt fest, welche Sichtattribute dieser
Knoten an seine Ausgangsknoten weitergibt. Bei einem Tabellenknoten sind die
Sichtattribute gleichbedeutend mit den Spalten der Tabelle. Die Sichtattribute
des Abfrageknotens sind die Attribute, welche im Ergebnis der Abfrage enthalten
sind. Damit Sichtattribute umbenannt werden können, wird jedem Eingangsknoten
eine Liste an Mappings zugeordnet. Hat der Knoten $N$ einen Eingangsknoten $E$
mit einem Mapping, dann bildet dieses ein Sichtattribut von $E$ auf ein
Sichtattribut von $N$ ab.

Die verschiedenen Sichtknotentypen sind für verschiedene Operationen zuständig.
Manche dieser Knoten haben noch zusätzliche Attribute, um die genaue Art und
Weise der Operation festzulegen.
Ein \foreignlanguage{english}{Join}-Knoten kann genutzt werden, um die
Ergebnisse der beiden Eingangsknoten zu verbinden. 

% section Modelle und Datenbankabfragen (end)

\section{Verschiedene Arten von Performance Analyse}
\label{sec:arten_performance_analyse}
\subsubsection*{Benchmarking}
\subsubsection*{Profiling}

% section Verschiedene Arten von Performance Analyse (end)

\section{Notwendige Grundlagen aus der Graphentheorie}
\label{sec:grundlagen_graphentheorie}

\autoref{def:gerichteter_graph} gibt eine Definition für einen gerichteten
Graphen \autocite[vgl.][S.220]{AlgorithmenUndDatenstrukturen}.
\begin{definition}
    Ein gerichteter Graph ist ein Tupel $G=(V,E)$. $V$ heißt Menge der Knoten.
    $E$ heißt Menge der gerichteten Kanten. Es gilt $V\ne\emptyset$ und $E \subset V \times
    V \setminus \{(v,v) | v \in V\}$. Zwischen zwei Knoten $u,v \in V$ gibt es
    eine Kante mit dem Anfangspunkt $u$ und dem Endpunkt $v$, wenn $(u,v) \in E$.
    \label{def:gerichteter_graph}
\end{definition}

\autoref{def:pfad} gibt eine Definition für einen Pfad in einem Graphen
\autocite[vgl.][S.221f]{AlgorithmenUndDatenstrukturen}.
\begin{definition}
    Ein Pfad $P$ in $G$  ist eine Folge von Knoten $v_0, \dots ,v_n$. Dabei
    muss $\forall i \in \{0; \dots; n-1 \} : (v_i,v_{i+1}) \in E$ gelten. $v_0$
    heißt Anfangspunkt von $P$, $v_n$ heißt Endpunkt von $P$ und $n$ heißt
    Länge von $P$.
    \label{def:pfad}
\end{definition}

\autoref{def:zyklen} gibt  eine Definition für Zyklen in einem gerichteten
Graphen und definiert den Begriff des azyklischen Graphen
\autocite[vgl.][S.222]{AlgorithmenUndDatenstrukturen}.
\begin{definition}
    Ein Pfad heißt geschlossen, wenn $v_0 = v_n$. Ein
    geschlossener Pfad heißt einfach, wenn $\forall i,j \in \{0; \dots; n-1\}, i
    \ne j:v_i \ne v_j$ gilt. In einem gerichteten Graphen heißt ein einfach
    geschlossener Pfad mit $n\geq 2$ auch Zyklus. Ein Graph heißt azyklisch,
    wenn er keine Zyklen besitzt.
    \label{def:zyklen}
\end{definition}

\autoref{def:gewichteter_graph} definiert den Begriff des gewichteten Graphen
\autocite[vgl.][S.253]{AlgorithmenUndDatenstrukturen}.
\begin{definition}
    $G$ heißt gewichtet, wenn es eine Abbildung $g:E\rightarrow \mathbb{R}$
    gibt, welche jeder Kante ein Gewicht zuordnet. Für $e\in E$ heißt $g(e)$
    Gewicht von $e$.
    \label{def:gewichteter_graph}
\end{definition}

\autoref{def:quelle_senke} definiert die Begriffe Quelle und Senke
\autocite[vgl.][S.306]{AlgorithmenUndDatenstrukturen}.
\begin{definition}
    Ein Knoten $q \in V$ heißt Quelle, wenn $ \neg \exists p \in V: (p,q) \in E
    $, $q$ also nach \autoref{def:kind_eltern} keine Elternknoten hat.
    $s \in V$ heißt Senke, wenn $ \neg \exists p \in V: (s,p) \in E $, $s$ also
    nach \autoref{def:kind_eltern} keine Kindknoten hat.
    \label{def:quelle_senke}
\end{definition}

\autoref{def:kind_eltern} definiert die Begriffe Kindknoten, Elternknoten und
Subknoten.
\begin{definition}
    Für zwei Knoten $c$ und $p$ gilt, $c$ ist Kindknoten von $p$, wenn $(p,c)\in E$.
    Umgekehrt gilt, wenn $(p,c)\in E$, $p$ ist Elternknoten von $c$.
    $c$ ist Subknoten von $p$, wenn $\exists P: (v_0=p) \land (v_n=c)$, es also
    einen Pfad von $p$ nach $c$ gibt.
    \label{def:kind_eltern}
\end{definition}
% section Notwendige Grundlagen aus der Graphentheorie (end)

\section{Statistische Grundlagen}
\label{sec:statistische_grundlagen}
% Metrisches Merkmal
In \autoref{def:metrisches_merkmal} wird der Begriff metrisches Merkmal
definiert  \autocite[vgl.][S.24]{Statistik}.
\begin{definition}
    Bei einem metrisch skaliertem Merkmale werden einzelne Ausprägungen anhand
    einer Zahlenskala bewertet, häufig in Verbindung mit einer Maßeinheit. Sie
    lassen sich anhand der Größe ordnen und vergleichen. Außerdem kann man die
    Abstände zwischen verschiedenen Ausprägungen messen und interpretieren.
    \label{def:metrisches_merkmal}
\end{definition}

% Arithmetisches Mittel
\autoref{def:durchschnitt} definiert die, im Kontext dieser Arbeit gleichbedeutenden, Begriffe
Durchschnitt, Mittelwert und Arithmetisches Mittel \autocite[vgl.][S.52f]{Statistik}.
\begin{definition}
    Das arithmetische Mittel von $n$ metrisch skalierten Beobachtungswerten
    $x_1,\dots,x_n$, genannt $\bar{x}$. 

    Es gilt: $\bar{x} = \frac{1}{n} \displaystyle\sum^{n}_{i=1}x_i$

    \label{def:durchschnitt}
\end{definition}

% Varianz
\autoref{def:varianz} beschreibt den Begriff der Varianz
\autocite[vgl.][S.399]{CorrelationBetweenRelatives}.
\begin{definition}
    Die empirische Varianz von $n$ metrisch skalierten Beobachtungswerten $x_1,
    \dots, x_n$, genannt $\tilde{s}^2$, ist der Durchschnitt der quadratischen
    Abweichung von $x_i$ von $\bar{x}$.

    Es gilt: $\tilde{s}^2 = \frac{1}{n} \displaystyle\sum^{n}_{i=1}(x_i - \bar{x})^2 $
    \label{def:varianz}
\end{definition}

% Standartabweichung
\autoref{def:standardabweichung} beschreibt den Begriff der Standardabweichung
\autocite[vgl.][S.80]{ContributionsToTheMathematicalTheory}.
\begin{definition}
    Die empirische Standardabweichung von $n$ metrisch skalierten Beobachtungswerten $x_1,
    \dots, x_n$, genannt $\tilde{s}$, ist die Quadratwurzel der Varianz.

    Es gilt: $\tilde{s} = \sqrt{\tilde{s}^2} = \sqrt{\frac{1}{n}
    \displaystyle\sum^{n}_{i=1}(x_i - \bar{x})^2} $
    \label{def:standardabweichung}
\end{definition}

% ?Regression?
% section Statistische Grundlagen (end)
