\chapter{Benchmarking-Konzept}
\section{Messung} % (fold)
\label{sec:Messung}
Um Messwerte sinnvoll miteinander vergleichen zu können, dürfen diese sich bis
auf einen Parameter nicht unterscheiden. Bei der Analyse, wie sich verschiedene
Modelle auf die Laufzeit der Optimierung von diesen auswirken, sind die Größe
und der Aufbau des Modells als Parameter denkbar. Der Aufbau, im folgenden auch
Art des Modells genannt, wird durch eine Funktion beschrieben, welche für einen
Parameter $n$ ein Modell der Größe $n$ erzeugt. $n$ ist dabei nicht zwingend
die Anzahl der Knoten im Modell, sondern könnte auch festlegen, wie
oft ein bestimmtes Muster in dem Modell wiederholt wurde.

Ist die Größe in einer Messreihe der veränderliche Parameter, muss der Aufbau
des Modells unveränderlich sein.
% section Messung (end)
\section{Testdaten} % (fold)
\label{sec:Testdaten}

% section Testdaten (end)
