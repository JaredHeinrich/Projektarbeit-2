\chapter{Grundlagen}

\section{Performance Analyse}
\label{sec:performance_analyse}

\subsubsection*{Benchmarking}


\subsubsection*{Profiling}

% section Verschiedene Arten von Performance Analyse (end)

\section{Notwendige Grundlagen aus der Graphentheorie}
\label{sec:grundlagen_graphentheorie}

\autoref{def:gerichteter_graph} gibt eine Definition für einen gerichteten
Graphen \autocite[vgl.][220]{AlgorithmenUndDatenstrukturen}.
\begin{definition}
    Ein gerichteter Graph ist ein Tupel $G=(V,E)$. $V$ heißt Menge der Knoten.
    $E$ heißt Menge der gerichteten Kanten. Es gilt $V\ne\emptyset$ und $E \subset V \times
    V \setminus \{(v,v) | v \in V\}$. Zwischen zwei Knoten $u,v \in V$ gibt es
    eine Kante mit dem Anfangspunkt $u$ und dem Endpunkt $v$, wenn $(u,v) \in E$.
    \label{def:gerichteter_graph}
\end{definition}

\autoref{def:pfad} gibt eine Definition für einen Pfad in einem Graphen
\autocite[vgl.][221f]{AlgorithmenUndDatenstrukturen}.
\begin{definition}
    Ein Pfad $P$ in $G$  ist eine Folge von Knoten $v_0, \dots ,v_n$. Dabei
    muss $\forall i \in \{0; \dots; n-1 \} : (v_i,v_{i+1}) \in E$ gelten. $v_0$
    heißt Anfangspunkt von $P$, $v_n$ heißt Endpunkt von $P$ und $n$ heißt
    Länge von $P$.
    \label{def:pfad}
\end{definition}

\autoref{def:zyklen} gibt  eine Definition für Zyklen in einem gerichteten
Graphen und definiert den Begriff des azyklischen Graphen
\autocite[vgl.][222]{AlgorithmenUndDatenstrukturen}.
\begin{definition}
    Ein Pfad heißt geschlossen, wenn $v_0 = v_n$. Ein
    geschlossener Pfad heißt einfach, wenn $\forall i,j \in \{0; \dots; n-1\}, i
    \ne j:v_i \ne v_j$ gilt. In einem gerichteten Graphen heißt ein einfach
    geschlossener Pfad mit $n\geq 2$ auch Zyklus. Ein Graph heißt azyklisch,
    wenn er keine Zyklen besitzt.
    \label{def:zyklen}
\end{definition}

\autoref{def:gewichteter_graph} definiert den Begriff des gewichteten Graphen
\autocite[vgl.][253]{AlgorithmenUndDatenstrukturen}.
\begin{definition}
    $G$ heißt gewichtet, wenn es eine Abbildung $g:E\rightarrow \mathbb{R}$
    gibt, welche jeder Kante ein Gewicht zuordnet. Für $e\in E$ heißt $g(e)$
    Gewicht von $e$.
    \label{def:gewichteter_graph}
\end{definition}

\autoref{def:quelle_senke} definiert die Begriffe Quelle und Senke
\autocite[vgl.][306]{AlgorithmenUndDatenstrukturen}.
\begin{definition}
    Ein Knoten $q \in V$ heißt Quelle, wenn $ \forall p \in V: (p,q) \not \in E
    $, $q$ also nach \autoref{def:kind_eltern} keine Elternknoten hat.
    $s \in V$ heißt Senke, wenn $ \forall p \in V: (s,p) \not \in E $, $s$ also
    nach \autoref{def:kind_eltern} keine Kindknoten hat.
    \label{def:quelle_senke}
\end{definition}

\autoref{def:kind_eltern} definiert die Begriffe Kindknoten, Elternknoten und
Subknoten.
\begin{definition}
    Für zwei Knoten $c$ und $p$ gilt, $c$ ist Kindknoten von $p$, wenn $(p,c)\in E$.
    Umgekehrt gilt, wenn $(p,c)\in E$, $p$ ist Elternknoten von $c$.
    $c$ ist Subknoten von $p$, wenn $\exists P: (v_0=p) \land (v_n=c)$, es also
    einen Pfad von $p$ nach $c$ gibt.
    \label{def:kind_eltern}
\end{definition}
% section Notwendige Grundlagen aus der Graphentheorie (end)
