\chapter{Konzept}

Für die Untersuchung des Optimierungsalgorithmus wird ein experimenteller
Ansatz statt einem analytischem gewählt, da die experimentelle Vorgehensweise
es zum einen einfacher macht sehr komplexe Algorithmen zu untersuchen, zum
anderen, eine experimentelle Untersuchung realitätsnähere Ergebnisse liefern
kann \autocite[vgl.][3]{ExperimentalMethods}. Hierzu werden in dieser Arbeit
zwei unabhängige Variablen betrachtet. Zum einen die Größe und zum anderen der
Aufbau des zu optimierenden Modells, diese wurden gewählt, da sie direkt
kontrolliert werden können und erwartet wird, dass sie einen großen Einfluss
auf die Laufzeit haben \autocite[vgl.][506]{ExperimentalAnalysis}. Der Aufbau
wird in dieser Arbeit als Art des Modells bezeichnet. Die gemessene abhängige
Variable ist die Laufzeit des Optimierungsvorgangs. Unabhängige Variablen sind
Variablen, welche aktiv verändert werden, während abhängige Variablen gemessen
werden \autocite[vgl.][236]{EmpirischeMethoden}. Damit Messergebnisse
sinnvoll vergleichen werden können, darf zwischen zwei Messungen nur eine
unabhängige Variable verändern werden. \autocite[vgl.][236]{EmpirischeMethoden}.
Deshalb werden mehrere Messreihen durchgeführt, wobei innerhalb einer Messreihe
der Art konstant ist, die Größe jedoch variabel ist. Für jede Art wird eine
neue Messreihe begonnen. Störvariablen, also Einflussfaktoren, welche ebenfalls
die abhängigen Variablen beeinflussen, jedoch während der Messung
unkontrolliert auftreten, ist \zB die Prozessorauslastung des Rechners, auch
welchem die Messung durchgeführt wird \autocite[vgl.][237]{EmpirischeMethoden}.
Ist der Prozessor weniger ausgelastet, dann ist die gemessene Zeit vermutlich
geringer, als wenn der Prozessor stark ausgelastet ist.

Um den Einfluss der Störvariablen möglichst gering zu halten, wird jede Messung
$n$ mal wiederholt. Aus diesen Messwerten wird nun ein Konfidenzintervall
gebildet, welches mit der Wahrscheinlichkeit $1 - \alpha$ den tatsächlichen
Erwartungswert $\mu$ enthält. Dazu werden die Messwerte als eine T-verteilte
Zufallsvariable $X$ betrachtet, da $n$ aufgrund der Dauer einer Messung nicht
sehr groß gewählt werden kann. Die Varianz $\sigma^2$ von $X$ ist dabei
unbekannt und muss anhand der Stichprobe geschätzt werden, für diese Schätzung
gilt: $\hat{\sigma}^2 = S^2$ \autocite[vgl][528]{Statistik}. Für das
($1-\alpha$)-Konfidenzintervall ergibt sich deshalb nach
\autocite[vgl.][533]{Statistik}:

\begin{equation*}
    [\bar{X} - t_{n-1,1-\alpha/2}\sqrt{S^2/n}, \bar{X} +
    t_{n-1,1-\alpha/2}\sqrt{S^2/n}]
\end{equation*}

Dabei ist $\bar{X}$ der Mittelwert der Stichprobe und $S^2$ die korrigierte
Stichprobenvarianz \autocites[vgl.][59, 502]{Statistik}. 
\begin{equation*}
    \bar{x} = \frac{1}{n} \displaystyle\sum^{n}_{i=1}x_i 
\end{equation*}
\begin{equation*}
    S^2 = \frac{1}{n-1}\displaystyle\sum^{n}_{i=1}(x_i-\bar{X})^2
\end{equation*}

Für die verschiedenen Arten von Modellen, werden Modelle, von reellen Abfragen
betrachtet, um aus diesen, allgemeine Regeln festzulegen, mit welchen man
Modelle variabler Größe aber derselben Art erzeugen kann. Die Modelle werden
für die Messung künstlich erzeugt, um die unabhängigen Variablen gezielt
verändern zu können. Auf die reellen Abfragen wird dabei zurückgegriffen, um
die Relevanz der Messung, für reelle Szenarien zu gewährleisten.
\autocite[Vgl.][500f]{ExperimentalAnalysis}

Diese Messdaten werden genutzt, um festzustellen, wie sich die Laufzeit der
Optimierung für Modelle bestimmter Art bei steigender Größe verhalten. Dabei
ist besonders interessant, ob sich die Laufzeit zur Größe linear verhält,
beziehungsweise ob es stärker oder schwächer ansteigt. Für die Optimierung des
Algorithmus sind nun die Modellarten interessant, bei welchen die Laufzeit
stärker als linear ansteigt, da diese Modellarten ein besonders hohes Potenzial
haben lange Laufzeiten zu verursachen. Um genauer herauszufinden, in welchem
Teil des Optimierungsalgorithmus besonders viel Zeit benötigt wurde, werden
diese Modelle nochmals mithilfe des in \autoref{sec:arten_performance_analyse}
beschriebenen Profilings untersucht. Dazu werden mehrere Modelle dieser Art
optimiert, während der Profiler protokolliert, in welchen Methoden sich wie
lange aufgehalten wurde. Anschließend muss beurteilt werden, ob die Zeit,
welche in dieser Methode benötigt wird, erwartbar ist oder sie geringer sein
sollte. Beziehungsweise, ob es eine Möglichkeit gibt diesen Teil des
Algorithmus zu beschleunigen.
