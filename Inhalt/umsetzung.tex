\chapter{Umsetzung}

\section{Benchmarking}

Da sich die Arbeit auf die Optimierungsfunktion der \ac{CE} beschränkt, kann
sich auch bei den Messungen auf diese beschränkt werden. Um die Leistung einer
bestimmten Funktion zu untersuchen, eignet sich, das in
\autoref{sec:arten_performance_analyse} beschriebenen, Benchmarking. Somit sind
die Messungen präziser auf diese Funktion ausgerichtet und es wird Aufwand
gespart, welcher auftreten würde, wenn man die Untersuchungen anhand einer
HANA-Installation durchführen würde. Genauer wird das in der \ac{CE} verwendete
Benchmarking-Framework Google Benchmarks genutzt. Dieses bietet die Möglichkeit
eine Messung mit mehreren Parametern und mehreren Variationen von diesen
durchzuführen. Es kann \zB für zwei Parameter Größe und Art jeweils eine Menge,
$G$ und $A$, an Werten festgelegt werden, für welche eine
Messung ausgeführt werden soll.

\begin{equation*}
    G=\{2;4;8\}
    \qquad A=\{j;p\}
\end{equation*}


\begin{equation*}
    K = G \times A = \{(2,j);(4,j);(8,j);(2,p);(4,p);(8,p)\}
\end{equation*}

Das kartesische Produkt $K$ ist eine Menge von Tupeln $(g,a)$. Jedes Tupel
spiegelt eine Kombination von Parametern wider, für welche eine Messung
durchgeführt wird \autocite[vgl.][50]{Mengenlehre}. Für alle $(g,a) \in K$ wird
das Modell der Art $a$ und der Größe $g$ optimiert und die Dauer dieser
Optimierung gemessen. Dieses Modell wird im Folgendem als Modell $(g,a)$
bezeichnet.

Die Parameter sind folgendermaßen definiert. Für jede Art wird eine Funktion
definiert, welche ein Modell dieser Art zurückgibt. Die Größe ist ein Wert $n$,
welcher an diese Funktion übergeben wird. Was genau eine Größe von $n$ bedeutet
ist dabei von Art zu Art unterschiedlich. Es kann also sein, dass das Modell
$(4,j)$ mit Größe $4$ und Art $j$ mehr Knoten hat als das Modell $(4,p)$,
obwohl sie dieselbe Größe haben. Das spielt jedoch keine Rolle, da diese
Messungen lediglich dazu dienen, zu untersuchen, ob die Laufzeit linear oder
stärker anwächst. Es ist jedoch wichtig, dass die Anzahl der Knoten, mit der
Größe linear anwächst, da sonst die Kurven nicht mehr vergleichbar währen.
Zwei Modelle sind also von gleicher Art, wenn sie mit der gleichen
Vorgehensweise erzeugt wurden. \autoref{fig:bsp_modell_art} zeigt verschiedene
Modelle. Von diesen könnten \zB \autoref{bsp_modell_art_1} und
\autoref{bsp_modell_art_2} von der gleichen Art sein, da diese erzeugt wurden,
indem ein \foreignlanguage{english}{Aggregation}-Knoten, $n$
\foreignlanguage{english}{Projection}-Knoten und ein
\foreignlanguage{english}{Table}-Knoten hintereinander gehängt werden.
Dabei könnte $n$ die Größe sein und $n+2$ die Anzahl der Knoten.

\begin{figure}
    \begin{subfigure}[b]{0.3\textwidth}
        \begin{tikzpicture}[
    node distance = 1cm,
    request/.style = {rectangle, draw, blue, minimum width=2cm, minimum height=0.8cm},
    op_node/.style = {circle, draw, minimum size=1cm},
    table_node/.style = {circle, draw, magenta, minimum size=1cm},
    arrow/.style = {->, >=stealth}
]

% Nodes
\node[request] (0) at (0,4) {Abfrage};
\node[right=0.2cm of 0] {Aggregation};

\node[op_node] (1) at (0,2) {1};
\node[right=0.2cm of 1] {Projection};

\node[table_node] (2) at (0,0) {2};
\node[right=0.2cm of 2] {Table};

% Arrows
\draw[arrow] (0) -- (1);
\draw[arrow] (1) -- (2);

\end{tikzpicture}
 \caption{Beispiel
        1}\label{bsp_modell_art_1}
    \end{subfigure}
    \begin{subfigure}[b]{0.3\textwidth}
        \input{Bilder/tikz/modell_projektionen_2.tex} \caption{Beispiel
        2}\label{bsp_modell_art_2}
    \end{subfigure}
    \begin{subfigure}[b]{0.3\textwidth} \input{Bilder/tikz/modell_join.tex}
        \caption{Beispiel 3}\label{bsp_modell_art_3}
    \end{subfigure}
    \caption{Beispiel Modelle}\label{fig:bsp_modell_art}
\end{figure}

\begin{listing}
    <@\textcolor[HTML]{724BFF}{\texttt{BENCHMARK}}@><@\textcolor[HTML]{000000}{\texttt{(}}@><@\textcolor[HTML]{000000}{\texttt{BM\_Optimizer}}@><@\textcolor[HTML]{000000}{\texttt{)}}@>
<@\textcolor[HTML]{000000}{\texttt{}}@><@\textcolor[HTML]{000000}{\texttt{\ \ }}@><@\textcolor[HTML]{1041FF}{\texttt{->}}@><@\textcolor[HTML]{724BFF}{\texttt{Apply}}@><@\textcolor[HTML]{000000}{\texttt{(}}@><@\textcolor[HTML]{000000}{\texttt{BenchOptimizerArguments}}@><@\textcolor[HTML]{000000}{\texttt{)}}@>
<@\textcolor[HTML]{000000}{\texttt{}}@><@\textcolor[HTML]{000000}{\texttt{\ \ }}@><@\textcolor[HTML]{1041FF}{\texttt{->}}@><@\textcolor[HTML]{724BFF}{\texttt{Repetitions}}@><@\textcolor[HTML]{000000}{\texttt{(}}@><@\textcolor[HTML]{DE6F10}{\texttt{20}}@><@\textcolor[HTML]{000000}{\texttt{)}}@>
<@\textcolor[HTML]{000000}{\texttt{}}@><@\textcolor[HTML]{000000}{\texttt{\ \ }}@><@\textcolor[HTML]{1041FF}{\texttt{->}}@><@\textcolor[HTML]{724BFF}{\texttt{Unit}}@><@\textcolor[HTML]{000000}{\texttt{(}}@><@\textcolor[HTML]{1F8F42}{\texttt{benchmark}}@><@\textcolor[HTML]{1041FF}{\texttt{::}}@><@\textcolor[HTML]{000000}{\texttt{kMicrosecond}}@><@\textcolor[HTML]{000000}{\texttt{);}}@>
<@\textcolor[HTML]{000000}{\texttt{}}@>
\caption{My Listing}
\label{my_listing}
\end{listing}

\autoref{my_listing}

\todo{Beschreibung der untersuchten Arten (Codeausschnitte anpassen und evntl einfügen)}

\section{Profiling}
