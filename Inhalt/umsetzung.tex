\chapter{Implementation der Untersuchungsmethode}

\section{Benchmarking}

Da sich die Arbeit auf die Optimierungsfunktion der \ac{CE} beschränkt, kann
sich auch bei den Messungen auf diese beschränkt werden. Um die Leistung einer
bestimmten Funktion zu untersuchen, eignet sich, das in
\autoref{sec:performance_analyse} beschriebenen, Benchmarking. Somit sind
die Messungen präziser auf diese Funktion ausgerichtet und es wird Aufwand
gespart, welcher auftreten würde, wenn man die Untersuchungen anhand einer
HANA-Installation durchführen würde. Genauer wird das in der \ac{CE} verwendete
Benchmarking-Framework Google Benchmark genutzt. Dieses bietet die Möglichkeit
eine Messung mit mehreren Parametern und mehreren Variationen von diesen
durchzuführen. Die festgelegten unabhängigen Variablen, Größe und Art, werden
durch solche Parameter widergespiegelt. Wie in \autoref{sec:google_benchmark}
beschrieben, gibt Google Benchmark die Werte \textit{CPU} und \textit{Time} für
die Laufzeit des Hauptabschnitts zurück. Für die abhängige Variable Zeit wird
\textit{Time} als Wert gewählt, da dieser auch potenzielle Wartezeiten
innerhalb der Optimierungsfunktion beinhaltet. Für die beiden Parameter kann
jeweils eine Menge, \zB $G$ für Größe und $A$ für Art, an Werten festgelegt
werden, für welche Messungen durchgeführt werden sollen.

\begin{equation*}
    G=\{2;4;8\}\qquad
    A=\{j;p\}
\end{equation*}


\begin{equation*}
    K = G \times A = \{(2,j);(4,j);(8,j);(2,p);(4,p);(8,p)\}
\end{equation*}

Das kartesische Produkt $K$ ist eine Menge von Tupeln $(g,a)$. Jedes Tupel
spiegelt eine Kombination von Parametern wider, für welche eine Messung
durchgeführt wird \autocite[vgl.][50]{Mengenlehre}. Für alle $(g,a) \in K$ wird
das Modell der Art $a$ und der Größe $g$ optimiert und die Dauer dieser
Optimierung gemessen. Dieses Modell wird im Folgendem als Modell $(g,a)$
bezeichnet.

Die Parameter sind folgendermaßen definiert. Für jede Art wird eine Funktion
definiert, welche ein Modell dieser Art zurückgibt. Die Größe ist ein Wert $n$,
welcher an diese Funktion übergeben wird. Was genau eine Größe von $n$ bedeutet
ist dabei von Art zu Art unterschiedlich. Es kann also sein, dass das Modell
$(4,j)$ mit Größe $4$ und Art $j$ mehr Knoten hat als das Modell $(4,p)$,
obwohl sie dieselbe Größe haben. Das spielt jedoch keine Rolle, da diese
Messungen lediglich dazu dienen, zu untersuchen, wie sich die Laufzeit bei
wachsender Größe verhält.
Zwei Modelle sind also von gleicher Art, wenn sie mit der gleichen
Vorgehensweise erzeugt wurden. \autoref{fig:bsp_modell_art} zeigt verschiedene
Modelle. Von diesen können \zB \autoref{bsp_modell_art_1} und
\autoref{bsp_modell_art_2} von der gleichen Art sein, da diese beiden erzeugt wurden,
indem ein \foreignlanguage{english}{Aggregation}-Knoten, $n$
\foreignlanguage{english}{Projection}-Knoten und ein
\foreignlanguage{english}{Table}-Knoten hintereinander gehängt werden.
Dabei kann \zB $n$ die Größe sein und $n+2$ die Anzahl der Knoten.
\autoref{bsp_modell_art_3} wurde nicht auf dieser Weise erzeugt, daher ist es
nicht von der gleichen Art. 

\begin{figure}
    \center
    \begin{subfigure}[b]{0.3\textwidth}
        \begin{tikzpicture}[
    node distance = 1cm,
    request/.style = {rectangle, draw, blue, minimum width=2cm, minimum height=0.8cm},
    op_node/.style = {circle, draw, minimum size=1cm},
    table_node/.style = {circle, draw, magenta, minimum size=1cm},
    arrow/.style = {->, >=stealth}
]

% Nodes
\node[request] (0) at (0,4) {Abfrage};
\node[right=0.2cm of 0] {Aggregation};

\node[op_node] (1) at (0,2) {1};
\node[right=0.2cm of 1] {Projection};

\node[table_node] (2) at (0,0) {2};
\node[right=0.2cm of 2] {Table};

% Arrows
\draw[arrow] (0) -- (1);
\draw[arrow] (1) -- (2);

\end{tikzpicture}

        \caption{Beispiel 1}\label{bsp_modell_art_1}
    \end{subfigure}
    \begin{subfigure}[b]{0.3\textwidth}
        \begin{tikzpicture}[
    node distance = 1cm,
    request/.style = {rectangle, draw, blue, minimum width=2cm, minimum height=0.8cm},
    op_node/.style = {circle, draw, minimum size=1cm},
    table_node/.style = {circle, draw, magenta, minimum size=1cm},
    arrow/.style = {->, >=stealth}
]

% Nodes
\node[request] (0) at (0,4) {Abfrage};
\node[right=0.2cm of 0] {Aggregation};

\node[op_node] (1) at (0,2) {1};
\node[right=0.2cm of 1] {Projection};

\node[op_node] (2) at (0,0) {2};
\node[right=0.2cm of 2] {Projection};

\node[table_node] (3) at (0,-2) {3};
\node[right=0.2cm of 3] {Table};

% Arrows
\draw[arrow] (0) -- (1);
\draw[arrow] (1) -- (2);
\draw[arrow] (2) -- (3);

\end{tikzpicture}

        \caption{Beispiel 2}\label{bsp_modell_art_2}
    \end{subfigure}
    \begin{subfigure}[b]{0.3\textwidth} 
        \begin{tikzpicture}[
    node distance = 1cm,
    request/.style = {rectangle, draw, blue, minimum width=2cm, minimum height=0.8cm},
    op_node/.style = {circle, draw, minimum size=1cm},
    table_node/.style = {circle, draw, magenta, minimum size=1cm},
    arrow/.style = {->, >=stealth}
]

% Nodes
\node[request] (0) at (0,3) {Abfrage};
\node[right=0.2cm of 0] {Aggregation};

\node[op_node] (1) at (0,1) {1};
\node[below=0.2cm of 1] {Join};

\node[table_node] (2) at (-2,-1) {2};
\node[below=0.2cm of 2] {Table};

\node[table_node] (3) at (2,-1) {3};
\node[below=0.2cm of 3] {Table};

% Arrows
\draw[arrow] (0) -- (1);
\draw[arrow] (1) to[out=210,in=90] (2);
\draw[arrow] (1) to[out=-30,in=90] (3);

\end{tikzpicture}

        \caption{Beispiel 3}\label{bsp_modell_art_3}
    \end{subfigure}
    \caption{Modelle gleicher und unterschiedlicher Art}\label{fig:bsp_modell_art}
\end{figure}

\begin{listing}
    \lstinputlisting[]{./Quellcode/tex/benchmarkDefinition.tex}
\caption{Benchmark Definition}
\label{code:benchmark_definition}
\end{listing}

\autoref{code:benchmark_definition} zeigt die Definition des Benchmarks,
welcher genutzt wird, um die Messungen durchzuführen. \verb+BM_Optimizer+,
abgebildet in \autoref{code:benchmark_optimizer} ist die Funktion, welche den
Code des Benchmarks enthält. \verb+BenchOptimizerArguments+, abgebildet in
\autoref{code:benchmark_arguments} legt alle Parameterpaare fest, für welche
die Funktion des Benchmarks aufgerufen werden soll. \verb+Repetitions(20)+ legt
die Anzahl der Wiederholungen fest und sorgt dafür, dass die Funktion für jedes
dieser Paare 20-mal ausgeführt wird. \verb+Unit(benchmark::kMicrosecond)+ legt
fest, dass die Einheit der Ausgabe Mikrosekunden ist.

\begin{listing}
    \lstinputlisting[]{./Quellcode/tex/benchOptimizerArguments.tex}
\caption{Benchmark Parameter}
\label{code:benchmark_arguments}
\end{listing}

\autoref{code:benchmark_arguments} erzeugt das kartesische Produkt von einer
Menge von Arten und einer Menge von Größen, und fügt jedes Tupel den Parameter
Paaren des Benchmarks hinzu.

\begin{listing}
    \lstinputlisting[]{./Quellcode/tex/benchOptimizer.tex}
\caption{Benchmark Hauptlogik}
\label{code:benchmark_optimizer}
\end{listing}

\autoref{code:benchmark_optimizer} beinhaltet die tatsächliche Logik des
Benchmarks. Diese Funktion wird für jedes Parameterpaar $n$-mal aufgerufen,
wobei $n$ die Anzahl der Wiederholungen ist. Für die Zeitmessung wird dabei nur
der Teil innerhalb der Iterationsschleife (\verb+for (auto _ : state) {}+) beachtet. Diese
wird $i$-mal ausgeführt, wobei $i$ die Anzahl der Iterationen ist. $i$ wird
abhängig von der Dauer einer Iteration und der Varianz der Zeit über alle
Iterationen hinweg automatisch bestimmt. Da die \verb+optimize+ Funktion die
\verb+RuntimeModel+ Instanz bearbeitet ist es problematisch, wenn die Funktion
mehrmals mit derselben Instanz aufgerufen wird. Deshalb wird die
Modell-Instanz innerhalb der Iterationsschleife erzeugt. Um die hierfür
benötigte Zeit nicht zu messen, wird die Zeitmessung davor pausiert, und danach wieder
fortgesetzt.

\section{Profiling}

\begin{listing}
    \lstinputlisting[]{./Quellcode/tex/profilingDefinition.tex}
\caption{Profiling Definition}
\label{code:profiling_definition}
\end{listing}

Das Profiling wird ebenfalls in einem Benchmark durchgeführt, um auch hier die
Messung nur auf die \verb+optimize+ Funktion zu beschränken. Die Definition
von diesem ist in \autoref{code:profiling_definition} gegeben. Der Benchmark
wird einmal für das zu untersuchende Parameterpaar ausgeführt. In
\autoref{code:profiling_definition} wurde exemplarisch für die Größe \verb+2048+ und
für die Art \verb+modelCreationJoinWithFilter+ gewählt.

\begin{listing}
    \lstinputlisting[]{./Quellcode/tex/profilingOptimizer.tex}
\caption{Profiling Hauptlogik}
\label{code:profiling_optimizer}
\end{listing}

\autoref{code:profiling_optimizer} zeigt die Logik für das Profiling, diese
sehr ähnlich zur Benchmarking-Logik, deshalb werden nur die Unterschiede
dargelegt. Da der Benchmark für das Profiling nicht genutzt wird, um die
Laufzeit die Iterationsschleife zu untersuchen, wird hier auf das Pausieren der
Zeitmessung verzichtet. Stattdessen, wird das Profiling während der
Initialisierung des Modells pausiert. Nachdem alle Iterationen durchlaufen
wurden, werden die gesammelten Profiling-Informationen ausgegeben.

\section{Untersuchte Parameter}
Die Parameter, welche für die Messungen verwendete werden sind
bereits in \autoref{code:benchmark_arguments} dargestellt. In diesem Abschnitt
werden jedoch die verschiedenen Modellarten noch genauer erläutert.
Die Modelle für die Messungen wurde wie in \autoref{fig:modell_generierung}
dargestellt erzeugt. Betrachtet man nur das Submodell, so hat jedes genau eine
Quelle und mindestens eine Senke. Die Quelle ist der Startknoten des
Submodells. Eine der Senken ist der Endknoten des Submodells, alle übrigen
Senke müssen Tabellenknoten sein. Um mehrere Submodelle aneinander zu hängen,
wird der Startknoten des einen Submodells an den Endknoten des anderen
gehängt. Die Anzahl der Submodelle wird dabei durch den Parameter Größe
bestimmt. Der Aufbau eines Submodells wird von der Art des Modells bestimmt.


\begin{figure}[h]
    \begin{center}
        \begin{tikzpicture}[
    node distance = 1cm,
    request/.style = {rectangle, draw, blue, minimum width=2cm, minimum height=0.8cm},
    submodel/.style = {rounded rectangle, draw, minimum size=1cm},
    small_op_node/.style = {circle, draw, minimum size=0.3cm},
    small_table_node/.style = {circle, draw, magenta, minimum size=0.3cm},
    table_node/.style = {circle, draw, magenta, minimum size=1cm},
    arrow/.style = {->, >=stealth}
]

% Nodes
\node[request] (A) at (0,4) {Abfrage};
\node[right=0.2cm of A] {Aggregation};

\node[small_op_node] (X) at (0,2) {};
\node[small_op_node] (Y) at (1,1) {};
\node[small_table_node] (Z) at (-1,1) {};
\node[draw, rounded rectangle, inner sep=5mm, fit=(X) (Y) (Z)] (0) {};
\node[right=0.2cm of 0] {Submodell};

\node[submodel] (1) at (0,-1) {...};
\node[right=0.2cm of 1] {Submodell};

\node[table_node] (T) at (0,-3) {T};
\node[right=0.2cm of T] {Table};

% Arrows
\draw[arrow] (A) -- (0);
\draw[arrow] (0) -- (1);
\draw[arrow] (1) -- (T);

\draw[arrow] (X) to[out=-30,in=90] (Y);
\draw[arrow] (X) to[out=210,in=90] (Z);

\draw[arrow] (0.north) -- (X.north);
\draw[arrow] (Y.west) to[out=210, in=90] (0.south);

\end{tikzpicture}

    \end{center}
    \caption{Darstellung der Modellgenerierung}\label{fig:modell_generierung}
\end{figure}

\autoref{fig:modell_arten} zeigt den Aufbau der Submodelle, für die drei
bereits in \autoref{code:benchmark_arguments} gezeigten Modellarten.
Das erste Submodell, besteht aus einem
einzelnen \foreignlanguage{english}{Projection}-Knoten. Dieser Knoten ist
Start- und Endknoten des Submodells. Das zweite Submodell besteht aus
einem \foreignlanguage{english}{Join}-Knoten, einem
\foreignlanguage{english}{Table}-Knoten und einem
\foreignlanguage{english}{Projection}-Knoten. Der
\foreignlanguage{english}{Join}-Knoten ist der Startknoten des Submodells und
hat den \foreignlanguage{english}{Table}- und den
\foreignlanguage{english}{Projection}-Knoten als Eingangsknoten.
Der \foreignlanguage{english}{Projection}-Knoten ist der Endknoten des
Submodells. Das dritte Submodell besteht aus einem
\foreignlanguage{english}{Union}-Knoten, zwei
\foreignlanguage{english}{Table}-Knoten und einem
\foreignlanguage{english}{Projection}-Knoten. Der
\foreignlanguage{english}{Union}-Knoten ist der Startknoten des Submodells und
hat die drei übrigen Knoten als Eingangsknoten. Der
\foreignlanguage{english}{Projection}-Knoten ist der Endknoten des Submodells.

\begin{figure}
    \center
    \begin{subfigure}[b]{0.2\textwidth}
        \center
        \begin{tikzpicture}[
    node distance = 1cm,
    op_node/.style = {circle, draw, minimum size=1cm},
    table_node/.style = {circle, draw, magenta, minimum size=1cm},
    arrow/.style = {->, >=stealth}
]

\node[op_node] (0) at (0,0) {0};
\node[right=0.2cm of 0] {Proj};

\draw[arrow] (0,1) -- (0);
\draw[arrow] (0) -- (0,-1);
\end{tikzpicture}
 
        \caption{\foreignlanguage{english}{PROJECTION}}\label{modell_art_projection}
    \end{subfigure}
    \begin{subfigure}[b]{0.3\textwidth}
        \center
        \begin{tikzpicture}[
    node distance = 1cm,
    op_node/.style = {circle, draw, minimum size=1cm},
    table_node/.style = {circle, draw, magenta, minimum size=1cm},
    arrow/.style = {->, >=stealth}
]

\node[op_node] (0) at (0,0) {0};
\node[right=0.2cm of 0] {Join};

\node[table_node] (1) at (-1,-2) {1};

\node[op_node] (2) at (1,-2) {2};
\node[right=0.2cm of 2] {Proj};


\draw[arrow] (0,1) -- (0);
\draw[arrow] (0) to[out=210,in=90] (1);
\draw[arrow] (0) to[out=-30,in=90] (2);
\draw[arrow] (2) -- (1,-3);
\end{tikzpicture}
 
        \caption{\foreignlanguage{english}{JOIN}}\label{modell_art_join}
    \end{subfigure}
    \begin{subfigure}[b]{0.4\textwidth} 
        \center
        \begin{tikzpicture}[
    node distance = 1cm,
    op_node/.style = {circle, draw, minimum size=1cm},
    table_node/.style = {circle, draw, magenta, minimum size=1cm},
    arrow/.style = {->, >=stealth}
]

\node[op_node] (0) at (0,0) {0};
\node[right=0.2cm of 0] {Union};

\node[table_node] (1) at (-1.5,-2) {1};

\node[table_node] (2) at (0,-2) {2};

\node[op_node] (3) at (1.5,-2) {3};
\node[right=0.2cm of 3] {Proj};

\draw[arrow] (0,1) -- (0);
\draw[arrow] (0) to[out=210,in=90] (1);
\draw[arrow] (0) -- (2);
\draw[arrow] (0) to[out=-30,in=90] (3);
\draw[arrow] (3) -- (1.5,-3);
\end{tikzpicture}

        \caption{\foreignlanguage{english}{UNION}}\label{modell_art_union}
    \end{subfigure}
    \caption{Darstellung der verschiedenen Modellarten}\label{fig:modell_arten}
\end{figure}
