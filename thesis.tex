\documentclass[
	fontsize=12pt,           % Leitlinien sprechen von Schriftgröße 12.
	paper=A4,
	twoside=false,
	listof=totoc,            % Tabellen- und Abbildungsverzeichnis ins Inhaltsverzeichnis
	bibliography=totoc,      % Literaturverzeichnis ins Inhaltsverzeichnis aufnehmen
	titlepage,               % Titlepage-Umgebung anstatt \maketitle
	headsepline,             % horizontale Linie unter Kolumnentitel
	abstract,              % Überschrift einschalten, Abstract muss in {abstract}-Umgebung stehen
    numbers=noenddot,
]{scrreprt}                  % Verwendung von KOMA-Report
\usepackage[utf8]{inputenc}  % UTF8 Encoding einschalten
\usepackage[ngerman]{babel}  % Neue deutsche Rechtschreibung
\usepackage[T1]{fontenc}     % Ausgabe von westeuropäischen Zeichen (auch Umlaute)
\usepackage{microtype}       % Trennung von Wörtern wird besser umgesetzt
\usepackage{lmodern}         % Nicht-gerasterte Schriftarten (bei MikTeX erforderlich)
\usepackage{graphicx}        % Einbinden von Grafiken erlauben
\usepackage{wrapfig}         % Grafiken fließend im Text
\usepackage{setspace}        % Zeilenabstand \singlespacing, \onehalfspaceing, \doublespacing
\usepackage{fancyvrb}
\usepackage[
	%showframe,                % Ränder anzeigen lassen
	left=2.7cm, right=2.5cm,
	top=2.5cm,  bottom=2.5cm,
	includeheadfoot
]{geometry}                      % Seitenlayout einstellen
\usepackage{scrlayer-scrpage}    % Gestaltung von Fuß- und Kopfzeilen
\usepackage{acronym}             % Abkürzungen, Abkürzungsverzeichnis
\usepackage{titletoc}            % Anpassungen am Inhaltsverzeichnis
\contentsmargin{0.75cm}          % Abstand im Inhaltsverzeichnis zw. Punkt und Seitenzahl
\usepackage{newfloat}
\DeclareFloatingEnvironment[fileext=frm,placement={!ht},name=Code-Ausschnitt]{listing}
%\captionsetup[listing]{labelfont=bf}
\usepackage[                     % Klickbare Links (enth. auch "nameref", "url" Package)
  hidelinks,                     % Blende die "URL Boxen" aus.
  breaklinks=true                % Breche zu lange URLs am Zeilenende um
]{hyperref}
\usepackage[hypcap=true]{caption}% Anker Anpassung für Referenzen
\usepackage[labelformat=simple]{subcaption}
\renewcommand\thesubfigure{(\alph{subfigure})}
\urlstyle{same}                  % Aktuelle Schrift auch für URLs
% Anpassung von autoref für Gleichungen (ergänzt runde Klammern) und Algorithm.
% Anstatt "Listing" kann auch z.B. "Code-Ausschnitt" verwendet werden. Dies sollte
% jedoch synchron gehalten werden mit \lstlistingname (siehe weiter unten).
\addto\extrasngerman{%
	\def\equationautorefname~#1\null{Gleichung~(#1)\null}
	\def\lstnumberautorefname{Zeile}
	\def\lstlistingautorefname{Listing}
	\def\algorithmautorefname{Algorithmus}
	% Damit einheitlich "Abschnitt 1.2[.3]" verwendet wird und nicht "Unterabschnitt 1.2.3"
	\def\subsectionautorefname{Abschnitt}
}

% ---- Abstand verkleinern von der Überschrift 
\renewcommand*{\chapterheadstartvskip}{\vspace*{.5\baselineskip}}

% Hierdurch werden Schusterjungen und Hurenkinder vermieden, d.h. einzelne Wörter
% auf der nächsten Seite oder in einer einzigen Zeile.
% LaTeX kann diese dennoch erzeugen, falls das Layout ansonsten nicht umsetzbar ist.
% Diese Werte sind aber gute Startwerte.
\widowpenalty10000
\clubpenalty10000


% ---- Für das Quellenverzeichnis
\usepackage[
	backend = biber,                % Verweis auf biber
	language = auto,
	style = authoryear,                % Nummerierung der Quellen mit Zahlen
	sorting = nyt,                 % none = Sortierung nach der Erscheinung im Dokument
	sortcites = true,               % Sortiert die Quellen innerhalb eines cite-Befehls
	block = space,                  % Extra Leerzeichen zwischen Blocks
	hyperref = true,                % Links sind klickbar auch in der Quelle
	%backref = true,                % Referenz, auf den Text an die zitierte Stelle
	bibencoding = auto,
	giveninits = true,              % Vornamen werden abgekürzt
	doi=true,                      % DOI nicht anzeigen
	isbn=true,                     % ISBN nicht anzeigen
    alldates=short                  % Datum immer als DD.MM.YYYY anzeigen
]{biblatex}
\addbibresource{literatur.bib}
\setcounter{biburlnumpenalty}{3000}     % Umbruchgrenze für Zahlen
\setcounter{biburlucpenalty}{6000}      % Umbruchgrenze für Großbuchstaben
\setcounter{biburllcpenalty}{9000}      % Umbruchgrenze für Kleinbuchstaben
\setlength\bibitemsep{3mm}
\DeclareNameAlias{default}{family-given}  % Nachname vor dem Vornamen
\AtBeginBibliography{\renewcommand{\multinamedelim}{\addslash\space
}\renewcommand{\finalnamedelim}{\multinamedelim}}  % Schrägstrich zwischen den Autorennamen
\DefineBibliographyStrings{german}{
  urlseen = {Einsichtnahme:},                      % Ändern des Titels von "besucht am"
}
\usepackage[babel,german=quotes]{csquotes}         % Deutsche Anführungszeichen + Zitate

% ---- Für Mathevorlage
\usepackage{amsmath}    % Erweiterung vom Mathe-Satz
\usepackage{amssymb}    % Lädt amsfonts und weitere Symbole
\usepackage{MnSymbol}   % Für Symbole, die in amssymb nicht enthalten sind.


% ---- Für Quellcodevorlage
\usepackage{scrhack}                    % Hack zur Verw. von listings in KOMA-Script
\usepackage{listings}                   % Darstellung von Quellcode
\usepackage{xcolor}                     % Einfache Verwendung von Farben
\lstset{
    escapeinside={<@}{@>},
	literate=%
		{á}{{\'a}}1 {é}{{\'e}}1 {í}{{\'i}}1 {ó}{{\'o}}1 {ú}{{\'u}}1
		{Á}{{\'A}}1 {É}{{\'E}}1 {Í}{{\'I}}1 {Ó}{{\'O}}1 {Ú}{{\'U}}1
		{à}{{\`a}}1 {è}{{\`e}}1 {ì}{{\`i}}1 {ò}{{\`o}}1 {ù}{{\`u}}1
		{À}{{\`A}}1 {È}{{\'E}}1 {Ì}{{\`I}}1 {Ò}{{\`O}}1 {Ù}{{\`U}}1
		{ä}{{\"a}}1 {ë}{{\"e}}1 {ï}{{\"i}}1 {ö}{{\"o}}1 {ü}{{\"u}}1
		{Ä}{{\"A}}1 {Ë}{{\"E}}1 {Ï}{{\"I}}1 {Ö}{{\"O}}1 {Ü}{{\"U}}1
		{â}{{\^a}}1 {ê}{{\^e}}1 {î}{{\^i}}1 {ô}{{\^o}}1 {û}{{\^u}}1
		{Â}{{\^A}}1 {Ê}{{\^E}}1 {Î}{{\^I}}1 {Ô}{{\^O}}1 {Û}{{\^U}}1
		{œ}{{\oe}}1 {Œ}{{\OE}}1 {æ}{{\ae}}1 {Æ}{{\AE}}1 {ß}{{\ss}}1
		{ű}{{\H{u}}}1 {Ű}{{\H{U}}}1 {ő}{{\H{o}}}1 {Ő}{{\H{O}}}1
		{ç}{{\c c}}1 {Ç}{{\c C}}1 {ø}{{\o}}1 {å}{{\r a}}1 {Å}{{\r A}}1
		{€}{{\euro}}1 {£}{{\pounds}}1 {«}{{\guillemotleft}}1
		{»}{{\guillemotright}}1 {ñ}{{\~n}}1 {Ñ}{{\~N}}1 {¿}{{?`}}1,
	breaklines=true,
	breakatwhitespace=true,
	numbers=left,
	basicstyle={\linespread{0.9}\small\ttfamily},
	numberstyle={\scriptsize\ttfamily\color{black!60}}, % the style for line numbers
	showstringspaces=false,
	xleftmargin=5.0ex,
	aboveskip=5mm,
}

% ---- Tabellen
\usepackage{booktabs}  % Für schönere Tabellen. Enthält neue Befehle wie \midrule
\usepackage{multirow}  % Mehrzeilige Tabellen
\usepackage{siunitx}   % Für SI Einheiten und das Ausrichten Nachkommastellen
\sisetup{locale=DE, range-phrase={~bis~}, output-decimal-marker={,}} % Damit ein Komma und kein Punkt verwendet wird.
\usepackage{xfrac} % Für siunitx Option "fraction-function=\sfrac"

% ---- Für Definitionsboxen in der Einleitung
\usepackage{amsthm}                     % Liefert die Grundlagen für Theoreme
\usepackage[framemethod=tikz]{mdframed} % Boxen für die Umrandung
% ------ Definition zum Strich vor eines Texts
\newmdtheoremenv[
  hidealllines = true,       % Rahmen komplett ausblenden
  leftline = true,           % Linie links einschalten
  innertopmargin = 0pt,      % Abstand oben
  innerbottommargin = 4pt,   % Abstand unten
  innerrightmargin = 0pt,    % Abstand rechts
  linewidth = 3pt,           % Linienbreite
  linecolor = gray!40,       % Linienfarbe
]{defStrich}{Definition}     % Name der des formats "defStrich"

% ------ Definition zum Eck-Kasten um einen Text
\newmdtheoremenv[
  hidealllines = true,
  innertopmargin = 6pt,
  linecolor = gray!40,
  singleextra={              % Eck-Markierungen für die Definition
    \draw[line width=3pt,gray!50,line cap=rect] (O|-P) -- +(1cm,0pt);
    \draw[line width=3pt,gray!50,line cap=rect] (O|-P) -- +(0pt,-1cm);
    \draw[line width=3pt,gray!50,line cap=rect] (O-|P) -- +(-1cm,0pt);
    \draw[line width=3pt,gray!50,line cap=rect] (O-|P) -- +(0pt,1cm);
  }
]{defEckKasten}{Definition}  % Name der des formats "defEckKasten"

\newmdtheoremenv[
  hidealllines = true,
  leftline = true,
  innertopmargin = 0pt,
  innerbottommargin = 4pt,
  innerrightmargin = 0pt,
  linewidth = 3pt,
  linecolor = gray!40,
  ]{definition}{Definition}[]
\newcommand{\definitionautorefname}{Definition}
  % Weitere Details sind ausgelagert

% ---- Für Todo Notes
\usepackage[disable]{todonotes}
%\usepackage{todonotes}
\setlength {\marginparwidth }{2cm}

% ---- Zum Einbinden von PDF-Dokumenten
\usepackage{pdfpages}

% ---- Für Tikz
\usepackage{tikz}
\usetikzlibrary{shapes,arrows,fit,positioning}


% ---- Elektronische Version oder Gedruckte Version?
% ---- Unterschied: Die elektronische Version enthält keinen Platzhalter für die Unterschrift
\usepackage{ifthen}
\newboolean{e-Abgabe}
\setboolean{e-Abgabe}{false}    % false=gedruckte Fassung

% ---- Persönlichen Daten:
\newcommand{\titel}{Performance Messung und Optimierung des Optimierungsalgorithmus für Datenbank-Abfragen in der HANA-Analytics-CalcEngine}
\newcommand{\titelheader}{Performance Messung und Optimierung in der HANA-Analytics-CalcEngine}
\newcommand{\arbeit}{Projektarbeit 2}
\newcommand{\studiengang}{Wirtschafts-Informatik}
\newcommand{\studienjahr}{2022}
\newcommand{\autor}{Jared Heinrich}
\newcommand{\autorReverse}{Heinrich, Jared}
\newcommand{\verfassungsort}{Mannheim}
\newcommand{\matrikelnr}{5101479}
\newcommand{\kurs}{WWI22SEA}
\newcommand{\bearbeitungsmonat}{August 2024}
\newcommand{\abgabe}{26. August 2024}
\newcommand{\bearbeitungszeitraum}{06.05.2024 - 25.08.2024}
\newcommand{\firmaName}{SAP SE}
\newcommand{\firmaStrasse}{Dietmar-Hopp-Allee 16}
\newcommand{\firmaPlz}{69190 Walldorf, Deutschland}
\newcommand{\betreuerFirma}{Rainer Agelek}
\newcommand{\betreuerDhbw}{Henning Pagnia}

\input{Inhalt/00_Latex/kopfundFusszeile}

% ---- Hilfreiches
\newcommand{\zB}{z.\,B. }   % "z.B." mit kleinem Leeraum dazwischen (ohne wäre nicht korrekt)
\newcommand{\dash}{d.\,h. }

\newcommand{\code}[1]{\texttt{#1}} % Ist einfacher zu schreiben als ständig \texttt und erlaubt
                                   % Änderungen im Nachhinein, wenn man z.B. Inline-Code anders stylen möchte.
% ---- Silbentrennung (falls LaTeX defaults falsch / nicht gewünscht sind)
\hyphenation{HANA}         % anstatt HA-NA
\hyphenation{Graph-Script} % anstatt GraphS-cript

% ---- Beginn des Dokuments
\begin{document}
\setlength{\parindent}{0pt}              % Keine Paragraphen Einrückung.
                                         % Dafür haben wir den Abstand zwischen den Paragraphen.
\setcounter{secnumdepth}{2}              % Nummerierungstiefe fürs Inhaltsverzeichnis
\setcounter{tocdepth}{1}                 % Tiefe des Inhaltsverzeichnisses. Ggf. so anpassen,
                                         % dass das Verzeichnis auf eine Seite passt.
\sffamily                                % Serifenlose Schrift verwenden.

% ---- Vorspann
% ------ Titelseite
\singlespacing
\include{Inhalt/01_Standard/titelseite}  % Titelseite
\newcounter{savepage}
\pagenumbering{Roman}                    % Römische Seitenzahlen
\onehalfspacing

% ------ Erklärung, Sperrvermerk, Abstact
\chapter*{Ehrenwörtliche Erklärung}

Ich versichere hiermit, dass ich die vorliegende Arbeit mit dem Thema: 

\begin{quote}
	\textit{\titel}
\end{quote} 

selbstständig
verfasst und keine anderen als die angegebenen Quellen und Hilfsmittel benutzt habe.

\vspace{0.25cm}

Ich versichere zudem, dass die eingereichte elektronische Fassung mit der gedruckten Fassung übereinstimmt.

\vspace{1cm}

\verfassungsort, den \today \\[0.5cm]
\ifthenelse{\boolean{e-Abgabe}}
	{\underline{Gez. \autor}}
	{\makebox[6cm]{\hrulefill}}\\ 
\autorReverse

\include{Inhalt/01_Standard/sperrvermerk}
\include{Inhalt/02_Abstract/abstract-en}
\include{Inhalt/02_Abstract/abstract-de}

% ------ Inhaltsverzeichnis
\singlespacing
\tableofcontents

% ------ Verzeichnisse
\renewcommand*{\chapterpagestyle}{plain}
\pagestyle{plain}
\include{Inhalt/03_Verzeichnisse/formelgroessen}
\chapter*{Abkürzungsverzeichnis}
\addcontentsline{toc}{chapter}{Abkürzungsverzeichnis} % Hinzufügen zum Inhaltsverzeichnis 

\begin{acronym}[CE] % längstes Kürzel wird verw. für den Abstand zw. Kürzel u. Text

	% Alphabetisch selbst sortieren - nicht verwendete Kürzel rausnehmen!
	\acro{CE}{Calculation Engine}

\end{acronym}

\listoffigures                          % Erzeugen des Abbildungsverzeichnisses 
\listoftables                           % Erzeugen des Tabellenverzeichnisses
\renewcommand{\lstlistlistingname}{Quellcodeverzeichnis}
\lstlistoflistings                      % Erzeugen des Listenverzeichnisses
\setcounter{savepage}{\value{page}}


% ---- Inhalt der Arbeit
\cleardoublepage
\pagenumbering{arabic}                  % Arabische Seitenzahlen für den Hauptteil
\setlength{\parskip}{0.5\baselineskip}  % Abstand zwischen Absätzen
\rmfamily
\renewcommand*{\chapterpagestyle}{scrheadings}
\pagestyle{scrheadings}
\onehalfspacing
\include{Inhalt/04_Inhalt/einleitung}
\chapter{Grundlagen}
\section{\acl{CE}}
\section{Ausführung von Datenbankabfragen}
\label{sec:execution_of_db_queries}
Datenbankabfragen werden in der \todo{Wo genau werden die Abfragen ausgeführt?
Was gehört alles zur CalcEngine?} \ac{CE} in drei Schritten ausgeführt. Zuerst
wird die Calculation View instanziiert. Anschließend wird dieses mithilfe
verschiedener Methoden optimiert. Dieses optimierte Modell wird dann auf der
Datenbank ausgeführt.
\section{Verschiedene Arten von Performance Analyse}
\label{sec:types_of_perf_analysis}
\subsection*{Benchmarking}
\subsection*{Profiling}

\chapter{Aktueller Stand der Performance Analyse in der \acl{CE}}
In der \ac{CE} werden bereits verschiedene Wege genutzt, um die Performance und
das Verhalten der HANA Datenbank zu analysieren. 
\section{Google Benchmark}
\label{sec:google_benchmark}

Google Benchmark ist ein Open Source Benchmarking Tool von Google,
welches es einem ermöglicht einzelne Funktionen in C++ zu benchmarken. 
Dazu kann man ähnlich zu den meisten Test Frameworks drei verschiedene
Codeabschnitte definieren.

Der Hauptabschnitt definiert was genau im Benchmark untersucht werden
soll. Diese legt die für die Messung relevante Logik fest.

Der Setup Abschnitt wird einmal vor jeder Ausführung des Benchmarks
aufgerufen. In dieser werden die Voraussetzungen für die Ausführung des Hauptabschnitts
geschaffen. Man könnte sie beispielsweise nutzen, um Testdaten für den Benchmark
zu generieren oder zu laden, da er nicht bei den Messungen beachtet wird.

Der Teardown Abschnitt wird nach jeder Ausführung des Benchmarks aufgerufen.
Dieser beeinflusst, wie der Setup Abschnitt, die Messung nicht.

Des Weiteren bietet Google Benchmark die Option, einen Benchmark mehrmals mit
unterschiedlichen Parametern durchzuführen. Um beispielsweise die Auswirkung
der Größe des Testdatensatzes auf das Ergebnis zu beobachten.

\autoref{fig:google_benchmark_ausgabe} zeigt eine Beispielhafte Google
Benchmark Ausgabe. Benchmark ist dabei der Name des Benchmarks und hinter dem
Schrägstrich die Größe des Inputs. Time ist die durchschnittliche Dauer einer
Ausführung des Hauptabschnitts über alle Iterationen hinweg. CPU ist die
durchschnittliche CPU Zeit einer Ausführung des Hauptabschnitts über alle
Iterationen hinweg. Iterations ist die Anzahl der durchgeführten Iterationen
des Hauptabschnitts, bis ein stabiler Durchschnittswert erreicht wurde.
\autocite[Vgl.][]{GoogleBenchmark}

\begin{figure}[h]
    \begin{center}
        \begin{verbatim}

---------------------------------------------------------------
Benchmark                          Time         CPU Iterations 
---------------------------------------------------------------
BM_RemoveDetachedNodes/2           4 us        4 us     189823 
BM_RemoveDetachedNodes/8           9 us        9 us      78165 
BM_RemoveDetachedNodes/64         58 us       58 us      12106 
BM_RemoveDetachedNodes/512       469 us      468 us       1515 
BM_RemoveDetachedNodes/4096     4237 us     4234 us        166 
BM_RemoveDetachedNodes/8192    10026 us    10019 us         60 
        \end{verbatim}
    \end{center}
    \caption{Google Benchmark Ausgabe}\label{fig:google_benchmark_ausgabe}
\end{figure}

Google Benchmark wird in der \ac{CE} meistens genutzt, um das Verhalten der
Laufzeit bestimmter Funktionen in künstlich generierten Testfällen zu
vergleichen. Hierbei wird jedoch keine HANA Instanz benötigt, da
keine Operationen auf einer Datenbank ausgeführt werden. Es werden folglich
auch keine Testdatensätze für die Datenbank benötigt, sondern nur Daten, auf
welchen man die zu analysierende Methode aufrufen kann. Solche Performance
Tests sind also im Vergleich zu Tests, welche auf eine HANA Instanz mit
Testdatensätzen zugreifen, relativ einfach. Da sie sich weniger Voraussetzungen
haben, und nur auf einen kleinen Teil der Logik fokussiert sind.

% section Google Benchmark (end)

\section{HANA Profiler}
\label{sec:hana_profiler}


Eine weitere Methode die Performance und das Verhalten von Software zu
analysieren ist, das bereits in \autoref{sec:arten_performance_analyse}
beschriebene, Profiling. In der \ac{CE} wird ein eigener Profiler verwendet.
Auf diesen kann entweder manuell oder im Code zugegriffen werden. Dabei sind
die wichtigsten Befehle \texttt{profiler clear} um die aktuellen
Profilinginformationen zurückzusetzen,  \texttt{profiler start} um den Profiler
zu starten, \texttt{profiler stop} um den Profiler zu stoppen und
\texttt{profiler print} um die gesammelten Profilinginformationen auszugeben.


Die Ausgabe erzeugt dabei ist dabei zwei gewichtete gerichtete azyklische
Graphen, nach \autoref{sec:grundlagen_graphentheorie}, von denen einer die Verteilung der CPU-Zeit und der andere die
Verteilung der Wartezeit, der aufgerufenen Funktionen beinhaltet. Im Folgenden
werden die beiden Graphen CPU-Graph und Warte-Graph genannt.
\autoref{fig:beispielausgabe_hana_profiler} stellt eine mögliche Ausgabe des
HANA Profilers dar. Die folgende Beschreibung gilt für den CPU- als auch den
Warte-Graphen. Jeder Knoten des Graphen spiegelt eine zur Laufzeit des
Profilers aufgerufene Funktion wider. Jeder Knoten beinhaltet drei
Informationen. Den Namen der aufgerufenen Funktion, sowie den Wert $I$ und den
Wert $E$. $I$ ist der Anteil der Gesamtzeit, welcher von diesem Knoten und all
seinen Subknoten benötigt wurde. $E$ ist der Anteil der Gesamtzeit, welcher
von diesem Knoten benötigt wurde. Folglich gilt für alle Knoten $I\geq E$.
Die Kindknoten eines Knoten $K$ sind die Funktionen, welche von $K$ aufgerufen
wurden.
\begin{figure}[h]
    \begin{center}
        \includegraphics[page=1]{Bilder/pdf/profiler_output_example.pdf}
    \end{center}
    \caption{Beispielausgabe des HANA Profilers}\label{fig:beispielausgabe_hana_profiler}
\end{figure}

% section HANA Profiler (end)

\chapter{Konzept}
\section{Verglich von Möglichkeiten Benchmarks durchzuführen}

\chapter{Umsetzung}
\section{Implementation der neuen Methode}

\chapter{Umsetzung und Interpretation der Messwerte}
In diesem Kapitel werden die Messwerte aus der Umsetzung dargestellt und
statistisch analysiert. Für die Interpretation der Messwerte werden, wie in
\autoref{tab:zahlen_modell_arten} gezeigt, den verschiedenen Arten von Modellen
Zahlen zugeordnet. 
\begin{table}[h]
\centering
\begin{tabular}{lc}
\toprule
\textbf{Modellart} & \textbf{Zahl} \\
\midrule
PROJECTION & 0\\
JOIN & 1\\
UNION & 2\\
\bottomrule
\end{tabular}
\caption{Zuordnung von Modellart zu Zahl }
\label{tab:zahlen_modell_arten}
\end{table}

\section{Benchmarking Ergebnisse}
\autoref{fig:messungen_all} zeigt die durchschnittlichen Messwerte sowie dem
0,99-Konfidenzintervall für diese Messwerte, für alle Modellarten.
Es lässt sich erkennen, dass für Modellart 0 die Zeit sich annähernd linear zur
Größe verhält. Für die Modellarten 1 und 2 steigt die Steigung mit zunehmender Größe
immer weiter an. Deshalb wird die Optimierung für diese Arten mithilfe von
Profiling genauer betrachtet.

\begin{figure}[h]
    \begin{center}
        \resizebox{!}{8cm}{\includegraphics[page=1]{Bilder/pdf/plot_all_release.pdf}}
    \end{center}
\caption{Durchschnittliche Zeit mit 0,99-Konfidenzintervall}\label{fig:messungen_all}
\end{figure}

\section{Profiling Ergebnisse}
\autoref{fig:profiling_art_1} zeigt einen Ausschnitt aus der Profiler-Ausgabe
für das Modell der Art 1 und der Größe 2048. Dieser Ausschnitt beinhaltet dabei
die relevantesten Knoten der Ausgabe. Der \verb+optimize+-Knoten
hat zwar noch weitere Kindknoten. Aufgrund der geringeren Auswirkung auf die
Laufzeit, wurden diese in der Grafik entfernt, um die Übersichtlichkeit zu
erhöhen. Die relevantesten von diesen Knoten werden jedoch im folgenden weiter
betrachtet. Die Ausgabe für Modellart 2 hat einen ähnlichen Aufbau.

\autoref{tab:e_values} stellt für ausgewählte Knoten den Anteil der von
Funktion dieses Knotens benötigten Zeit von der Gesamtzeit dar. Es sind jeweils
alle Messwerte für die verschiedenen Größen angegeben, um auch das Verhalten
analysieren zu können. Diese Werte entsprechen dem in
\autoref{sec:hana_profiler} erläutertem Wert $E$ aus der Profiler-Ausgabe.
Auffällig ist die \verb+opptimize+-Funktion, diese benötigt sowohl bei
Modellart 1 als auch bei Modellart 2 einen sehr großen Anteil der Zeit. Dieser
Anteil steigt jedoch zumindest bei Modellart 2 nicht eindeutig mit der Größe. Dies
bedeutet jedoch nicht, dass die benötigte Zeit in dieser Funktion ebenfalls
nicht steigt, da diese Angaben prozentual zu der jeweiligen Gesamtdauer sind.
Die anderen Knoten sind deshalb interessant, weil der Anteil an der Gesamtdauer
mit der Größe des Modells zunimmt. Die größte Zunahme von Prozentpunkten ist
dabei $32 \% - 19 \% = 13 \%$, die größte prozentuale Steigerung ist
$\frac{12\%}{1,5\%} - 1 = 700\%$. Aufgrund des hohen Anteils oder der großen
Steigerung von diesem bieten sich alle in \autoref{tab:e_values} dargestellten
Funktion für eine nähere Betrachtung an.

\begin{table}[h]
\centering
\resizebox{\textwidth}{!}{
\begin{tabular}{|l|l|c|c|c|c|}
\toprule
\textbf{Art} & \textbf{Funktion} & \textbf{2048} & \textbf{4096} & \textbf{8192} &\textbf{16384} \\
\midrule
1 & Optimizer2::optimize                           & 18 \% & 22 \% & 24 \% & 24 \% \\ \hline
1 & CombineJoinOverProjectionPattern::doesMatch    & 19 \% & 26 \% & 30 \% & 32 \% \\\hline
1 & BuildExpresionFilterPattern::doesMatchInternal & 1,5 \% & 6,1 \% & 11 \% & 12 \% \\\hline
2 & Optimizer2::optimize                           & 37 \% & 32 \% & 33 \% & 34 \% \\ \hline
2 & BuildExpresionFilterPattern::doesMatchInternal & 4,2 \% & 10 \% & 10 \% & 14 \% \\
\bottomrule
\end{tabular}
}
\caption{Anteil an der Gesamtlaufzeit ausgewählter Knoten}\label{tab:e_values}
\end{table}

\include{Inhalt/04_Inhalt/fazit}

% ---- Literaturverzeichnis
\cleardoublepage
\renewcommand*{\chapterpagestyle}{plain}
\pagestyle{plain}
\pagenumbering{Roman}                   % Römische Seitenzahlen
\setcounter{page}{\numexpr\value{savepage}+1}
\printbibliography[title=Literaturverzeichnis]

% ---- Anhang
\appendix
%\clearpage
%\pagenumbering{Roman}  % römische Seitenzahlen für Anhang

\newpage
\end{document}
