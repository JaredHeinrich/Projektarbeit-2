\documentclass[
	fontsize=12pt,           % Leitlinien sprechen von Schriftgröße 12.
	paper=A4,
	twoside=false,
	listof=totoc,            % Tabellen- und Abbildungsverzeichnis ins Inhaltsverzeichnis
	bibliography=totoc,      % Literaturverzeichnis ins Inhaltsverzeichnis aufnehmen
	titlepage,               % Titlepage-Umgebung anstatt \maketitle
	headsepline,             % horizontale Linie unter Kolumnentitel
	abstract,              % Überschrift einschalten, Abstract muss in {abstract}-Umgebung stehen
    numbers=noenddot,
]{scrreprt}                  % Verwendung von KOMA-Report
\usepackage[utf8]{inputenc}  % UTF8 Encoding einschalten
\usepackage[ngerman]{babel}  % Neue deutsche Rechtschreibung
\usepackage[T1]{fontenc}     % Ausgabe von westeuropäischen Zeichen (auch Umlaute)
\usepackage{microtype}       % Trennung von Wörtern wird besser umgesetzt
\usepackage{lmodern}         % Nicht-gerasterte Schriftarten (bei MikTeX erforderlich)
\usepackage{graphicx}        % Einbinden von Grafiken erlauben
\usepackage{wrapfig}         % Grafiken fließend im Text
\usepackage{setspace}        % Zeilenabstand \singlespacing, \onehalfspaceing, \doublespacing
\usepackage{fancyvrb}
\usepackage[
	%showframe,                % Ränder anzeigen lassen
	left=2.7cm, right=2.5cm,
	top=2.5cm,  bottom=2.5cm,
	includeheadfoot
]{geometry}                      % Seitenlayout einstellen
\usepackage{scrlayer-scrpage}    % Gestaltung von Fuß- und Kopfzeilen
\usepackage{acronym}             % Abkürzungen, Abkürzungsverzeichnis
\usepackage{titletoc}            % Anpassungen am Inhaltsverzeichnis
\contentsmargin{0.75cm}          % Abstand im Inhaltsverzeichnis zw. Punkt und Seitenzahl
\usepackage{newfloat}
\DeclareFloatingEnvironment[fileext=frm,placement={!ht},name=Code-Ausschnitt]{listing}
%\captionsetup[listing]{labelfont=bf}
\usepackage[                     % Klickbare Links (enth. auch "nameref", "url" Package)
  hidelinks,                     % Blende die "URL Boxen" aus.
  breaklinks=true                % Breche zu lange URLs am Zeilenende um
]{hyperref}
\usepackage[hypcap=true]{caption}% Anker Anpassung für Referenzen
\usepackage[labelformat=simple]{subcaption}
\renewcommand\thesubfigure{(\alph{subfigure})}
\urlstyle{same}                  % Aktuelle Schrift auch für URLs
% Anpassung von autoref für Gleichungen (ergänzt runde Klammern) und Algorithm.
% Anstatt "Listing" kann auch z.B. "Code-Ausschnitt" verwendet werden. Dies sollte
% jedoch synchron gehalten werden mit \lstlistingname (siehe weiter unten).
\addto\extrasngerman{%
	\def\equationautorefname~#1\null{Gleichung~(#1)\null}
	\def\lstnumberautorefname{Zeile}
	\def\lstlistingautorefname{Listing}
	\def\algorithmautorefname{Algorithmus}
	% Damit einheitlich "Abschnitt 1.2[.3]" verwendet wird und nicht "Unterabschnitt 1.2.3"
	\def\subsectionautorefname{Abschnitt}
}

% ---- Abstand verkleinern von der Überschrift 
\renewcommand*{\chapterheadstartvskip}{\vspace*{.5\baselineskip}}

% Hierdurch werden Schusterjungen und Hurenkinder vermieden, d.h. einzelne Wörter
% auf der nächsten Seite oder in einer einzigen Zeile.
% LaTeX kann diese dennoch erzeugen, falls das Layout ansonsten nicht umsetzbar ist.
% Diese Werte sind aber gute Startwerte.
\widowpenalty10000
\clubpenalty10000


% ---- Für das Quellenverzeichnis
\usepackage[
	backend = biber,                % Verweis auf biber
	language = auto,
	style = authoryear,                % Nummerierung der Quellen mit Zahlen
	sorting = nyt,                 % none = Sortierung nach der Erscheinung im Dokument
	sortcites = true,               % Sortiert die Quellen innerhalb eines cite-Befehls
	block = space,                  % Extra Leerzeichen zwischen Blocks
	hyperref = true,                % Links sind klickbar auch in der Quelle
	%backref = true,                % Referenz, auf den Text an die zitierte Stelle
	bibencoding = auto,
	giveninits = true,              % Vornamen werden abgekürzt
	doi=true,                      % DOI nicht anzeigen
	isbn=true,                     % ISBN nicht anzeigen
    alldates=short                  % Datum immer als DD.MM.YYYY anzeigen
]{biblatex}
\addbibresource{literatur.bib}
\setcounter{biburlnumpenalty}{3000}     % Umbruchgrenze für Zahlen
\setcounter{biburlucpenalty}{6000}      % Umbruchgrenze für Großbuchstaben
\setcounter{biburllcpenalty}{9000}      % Umbruchgrenze für Kleinbuchstaben
\setlength\bibitemsep{3mm}
\DeclareNameAlias{default}{family-given}  % Nachname vor dem Vornamen
\AtBeginBibliography{\renewcommand{\multinamedelim}{\addslash\space
}\renewcommand{\finalnamedelim}{\multinamedelim}}  % Schrägstrich zwischen den Autorennamen
\DefineBibliographyStrings{german}{
  urlseen = {Einsichtnahme:},                      % Ändern des Titels von "besucht am"
}
\usepackage[babel,german=quotes]{csquotes}         % Deutsche Anführungszeichen + Zitate

% ---- Für Mathevorlage
\usepackage{amsmath}    % Erweiterung vom Mathe-Satz
\usepackage{amssymb}    % Lädt amsfonts und weitere Symbole
\usepackage{MnSymbol}   % Für Symbole, die in amssymb nicht enthalten sind.


% ---- Für Quellcodevorlage
\usepackage{scrhack}                    % Hack zur Verw. von listings in KOMA-Script
\usepackage{listings}                   % Darstellung von Quellcode
\usepackage{xcolor}                     % Einfache Verwendung von Farben
\lstset{
    escapeinside={<@}{@>},
	literate=%
		{á}{{\'a}}1 {é}{{\'e}}1 {í}{{\'i}}1 {ó}{{\'o}}1 {ú}{{\'u}}1
		{Á}{{\'A}}1 {É}{{\'E}}1 {Í}{{\'I}}1 {Ó}{{\'O}}1 {Ú}{{\'U}}1
		{à}{{\`a}}1 {è}{{\`e}}1 {ì}{{\`i}}1 {ò}{{\`o}}1 {ù}{{\`u}}1
		{À}{{\`A}}1 {È}{{\'E}}1 {Ì}{{\`I}}1 {Ò}{{\`O}}1 {Ù}{{\`U}}1
		{ä}{{\"a}}1 {ë}{{\"e}}1 {ï}{{\"i}}1 {ö}{{\"o}}1 {ü}{{\"u}}1
		{Ä}{{\"A}}1 {Ë}{{\"E}}1 {Ï}{{\"I}}1 {Ö}{{\"O}}1 {Ü}{{\"U}}1
		{â}{{\^a}}1 {ê}{{\^e}}1 {î}{{\^i}}1 {ô}{{\^o}}1 {û}{{\^u}}1
		{Â}{{\^A}}1 {Ê}{{\^E}}1 {Î}{{\^I}}1 {Ô}{{\^O}}1 {Û}{{\^U}}1
		{œ}{{\oe}}1 {Œ}{{\OE}}1 {æ}{{\ae}}1 {Æ}{{\AE}}1 {ß}{{\ss}}1
		{ű}{{\H{u}}}1 {Ű}{{\H{U}}}1 {ő}{{\H{o}}}1 {Ő}{{\H{O}}}1
		{ç}{{\c c}}1 {Ç}{{\c C}}1 {ø}{{\o}}1 {å}{{\r a}}1 {Å}{{\r A}}1
		{€}{{\euro}}1 {£}{{\pounds}}1 {«}{{\guillemotleft}}1
		{»}{{\guillemotright}}1 {ñ}{{\~n}}1 {Ñ}{{\~N}}1 {¿}{{?`}}1,
	breaklines=true,
	breakatwhitespace=true,
	numbers=left,
	basicstyle={\linespread{0.9}\small\ttfamily},
	numberstyle={\scriptsize\ttfamily\color{black!60}}, % the style for line numbers
	showstringspaces=false,
	xleftmargin=5.0ex,
	aboveskip=5mm,
}

% ---- Tabellen
\usepackage{booktabs}  % Für schönere Tabellen. Enthält neue Befehle wie \midrule
\usepackage{multirow}  % Mehrzeilige Tabellen
\usepackage{siunitx}   % Für SI Einheiten und das Ausrichten Nachkommastellen
\sisetup{locale=DE, range-phrase={~bis~}, output-decimal-marker={,}} % Damit ein Komma und kein Punkt verwendet wird.
\usepackage{xfrac} % Für siunitx Option "fraction-function=\sfrac"

% ---- Für Definitionsboxen in der Einleitung
\usepackage{amsthm}                     % Liefert die Grundlagen für Theoreme
\usepackage[framemethod=tikz]{mdframed} % Boxen für die Umrandung
% ------ Definition zum Strich vor eines Texts
\newmdtheoremenv[
  hidealllines = true,       % Rahmen komplett ausblenden
  leftline = true,           % Linie links einschalten
  innertopmargin = 0pt,      % Abstand oben
  innerbottommargin = 4pt,   % Abstand unten
  innerrightmargin = 0pt,    % Abstand rechts
  linewidth = 3pt,           % Linienbreite
  linecolor = gray!40,       % Linienfarbe
]{defStrich}{Definition}     % Name der des formats "defStrich"

% ------ Definition zum Eck-Kasten um einen Text
\newmdtheoremenv[
  hidealllines = true,
  innertopmargin = 6pt,
  linecolor = gray!40,
  singleextra={              % Eck-Markierungen für die Definition
    \draw[line width=3pt,gray!50,line cap=rect] (O|-P) -- +(1cm,0pt);
    \draw[line width=3pt,gray!50,line cap=rect] (O|-P) -- +(0pt,-1cm);
    \draw[line width=3pt,gray!50,line cap=rect] (O-|P) -- +(-1cm,0pt);
    \draw[line width=3pt,gray!50,line cap=rect] (O-|P) -- +(0pt,1cm);
  }
]{defEckKasten}{Definition}  % Name der des formats "defEckKasten"

\newmdtheoremenv[
  hidealllines = true,
  leftline = true,
  innertopmargin = 0pt,
  innerbottommargin = 4pt,
  innerrightmargin = 0pt,
  linewidth = 3pt,
  linecolor = gray!40,
  ]{definition}{Definition}[]
\newcommand{\definitionautorefname}{Definition}
  % Weitere Details sind ausgelagert

% ---- Für Todo Notes
\usepackage[disable]{todonotes}
%\usepackage{todonotes}
\setlength {\marginparwidth }{2cm}

% ---- Zum Einbinden von PDF-Dokumenten
\usepackage{pdfpages}

% ---- Für Tikz
\usepackage{tikz}
\usetikzlibrary{shapes,arrows,fit,positioning}


% ---- Elektronische Version oder Gedruckte Version?
% ---- Unterschied: Die elektronische Version enthält keinen Platzhalter für die Unterschrift
\usepackage{ifthen}
\newboolean{e-Abgabe}
\setboolean{e-Abgabe}{false}    % false=gedruckte Fassung

% ---- Persönlichen Daten:
\newcommand{\titel}{Performance Analyse der Optimierung von Datenbankabfragen in der HANA \acl{CE}}
%\newcommand{\titelheader}{Performance Messung und Optimierung in der HANA-Analytics-CalcEngine}
\newcommand{\arbeit}{Projektarbeit 2}
\newcommand{\studiengang}{Wirtschaftsinformatik}
\newcommand{\studienjahr}{2022}
\newcommand{\autor}{Jared Heinrich}
\newcommand{\autorReverse}{Heinrich, Jared}
\newcommand{\verfassungsort}{Mannheim}
\newcommand{\matrikelnr}{5101479}
\newcommand{\kurs}{WWI22SEA}
\newcommand{\bearbeitungsmonat}{August 2024}
\newcommand{\abgabe}{26. August 2024}
\newcommand{\bearbeitungszeitraum}{06.05.2024 - 25.08.2024}
\newcommand{\firmaName}{SAP SE}
\newcommand{\firmaStrasse}{Dietmar-Hopp-Allee 16}
\newcommand{\firmaPlz}{69190 Walldorf, Deutschland}
\newcommand{\betreuerFirma}{Rainer Agelek}
\newcommand{\betreuerDhbw}{Prof. Dr. Hans-Henning Pagnia}

\input{Latex/kopfundFusszeile}

% ---- Hilfreiches
\newcommand{\zB}{z.\,B. }   % "z.B." mit kleinem Leeraum dazwischen (ohne wäre nicht korrekt)
\newcommand{\dash}{d.\,h. }

\newcommand{\code}[1]{\texttt{#1}} % Ist einfacher zu schreiben als ständig \texttt und erlaubt
                                   % Änderungen im Nachhinein, wenn man z.B. Inline-Code anders stylen möchte.
% ---- Silbentrennung (falls LaTeX defaults falsch / nicht gewünscht sind)
\hyphenation{HANA}         % anstatt HA-NA
\hyphenation{Graph-Script} % anstatt GraphS-cript

% ---- Beginn des Dokuments
\begin{document}
\setlength{\parindent}{0pt}              % Keine Paragraphen Einrückung.
                                         % Dafür haben wir den Abstand zwischen den Paragraphen.
\setcounter{secnumdepth}{2}              % Nummerierungstiefe fürs Inhaltsverzeichnis
\setcounter{tocdepth}{1}                 % Tiefe des Inhaltsverzeichnisses. Ggf. so anpassen,
                                         % dass das Verzeichnis auf eine Seite passt.
\sffamily                                % Serifenlose Schrift verwenden.

% ---- Vorspann
% ------ Titelseite
\singlespacing
\include{Standard/titelseite}  % Titelseite
\newcounter{savepage}
\pagenumbering{Roman}                    % Römische Seitenzahlen
\onehalfspacing

% ------ Erklärung, Sperrvermerk, Abstact
\chapter*{Ehrenwörtliche Erklärung}

Ich versichere hiermit, dass ich die vorliegende Arbeit mit dem Thema: 

\begin{quote}
	\textit{\titel}
\end{quote} 

selbstständig
verfasst und keine anderen als die angegebenen Quellen und Hilfsmittel benutzt habe.

\vspace{0.25cm}

Ich versichere zudem, dass die eingereichte elektronische Fassung mit der gedruckten Fassung übereinstimmt.

\vspace{1cm}

\verfassungsort, den \today \\[0.5cm]
\ifthenelse{\boolean{e-Abgabe}}
	{\underline{Gez. \autor}}
	{\makebox[6cm]{\hrulefill}}\\ 
\autorReverse

\include{Standard/sperrvermerk}
\include{Abstract/abstract-de}

% ------ Inhaltsverzeichnis
\singlespacing
\tableofcontents

% ------ Verzeichnisse
\renewcommand*{\chapterpagestyle}{plain}
\pagestyle{plain}
\include{Verzeichnisse/formelgroessen}
\chapter*{Abkürzungsverzeichnis}
\addcontentsline{toc}{chapter}{Abkürzungsverzeichnis} % Hinzufügen zum Inhaltsverzeichnis 

\begin{acronym}[CE] % längstes Kürzel wird verw. für den Abstand zw. Kürzel u. Text

	% Alphabetisch selbst sortieren - nicht verwendete Kürzel rausnehmen!
	\acro{CE}{Calculation Engine}

\end{acronym}

\listoffigures                          % Erzeugen des Abbildungsverzeichnisses 
\listoftables                           % Erzeugen des Tabellenverzeichnisses
\renewcommand{\listlistingname}{Quellcodeverzeichnis}
\listoflistings
\setcounter{savepage}{\value{page}}


% ---- Inhalt der Arbeit
\cleardoublepage
\pagenumbering{arabic}                  % Arabische Seitenzahlen für den Hauptteil
\setlength{\parskip}{0.5\baselineskip}  % Abstand zwischen Absätzen
\rmfamily
\renewcommand*{\chapterpagestyle}{scrheadings}
\pagestyle{scrheadings}
\onehalfspacing
\include{Inhalt/einleitung}
\chapter{Grundlagen}

\section{Performance Analyse}
\label{sec:performance_analyse}

\subsubsection*{Benchmarking}


\subsubsection*{Profiling}

% section Verschiedene Arten von Performance Analyse (end)

\section{Notwendige Grundlagen aus der Graphentheorie}
\label{sec:grundlagen_graphentheorie}

\autoref{def:gerichteter_graph} gibt eine Definition für einen gerichteten
Graphen \autocite[vgl.][220]{AlgorithmenUndDatenstrukturen}.
\begin{definition}
    Ein gerichteter Graph ist ein Tupel $G=(V,E)$. $V$ heißt Menge der Knoten.
    $E$ heißt Menge der gerichteten Kanten. Es gilt $V\ne\emptyset$ und $E \subset V \times
    V \setminus \{(v,v) | v \in V\}$. Zwischen zwei Knoten $u,v \in V$ gibt es
    eine Kante mit dem Anfangspunkt $u$ und dem Endpunkt $v$, wenn $(u,v) \in E$.
    \label{def:gerichteter_graph}
\end{definition}

\autoref{def:pfad} gibt eine Definition für einen Pfad in einem Graphen
\autocite[vgl.][221f]{AlgorithmenUndDatenstrukturen}.
\begin{definition}
    Ein Pfad $P$ in $G$  ist eine Folge von Knoten $v_0, \dots ,v_n$. Dabei
    muss $\forall i \in \{0; \dots; n-1 \} : (v_i,v_{i+1}) \in E$ gelten. $v_0$
    heißt Anfangspunkt von $P$, $v_n$ heißt Endpunkt von $P$ und $n$ heißt
    Länge von $P$.
    \label{def:pfad}
\end{definition}

\autoref{def:zyklen} gibt  eine Definition für Zyklen in einem gerichteten
Graphen und definiert den Begriff des azyklischen Graphen
\autocite[vgl.][222]{AlgorithmenUndDatenstrukturen}.
\begin{definition}
    Ein Pfad heißt geschlossen, wenn $v_0 = v_n$. Ein
    geschlossener Pfad heißt einfach, wenn $\forall i,j \in \{0; \dots; n-1\}, i
    \ne j:v_i \ne v_j$ gilt. In einem gerichteten Graphen heißt ein einfach
    geschlossener Pfad mit $n\geq 2$ auch Zyklus. Ein Graph heißt azyklisch,
    wenn er keine Zyklen besitzt.
    \label{def:zyklen}
\end{definition}

\autoref{def:gewichteter_graph} definiert den Begriff des gewichteten Graphen
\autocite[vgl.][253]{AlgorithmenUndDatenstrukturen}.
\begin{definition}
    $G$ heißt gewichtet, wenn es eine Abbildung $g:E\rightarrow \mathbb{R}$
    gibt, welche jeder Kante ein Gewicht zuordnet. Für $e\in E$ heißt $g(e)$
    Gewicht von $e$.
    \label{def:gewichteter_graph}
\end{definition}

\autoref{def:quelle_senke} definiert die Begriffe Quelle und Senke
\autocite[vgl.][306]{AlgorithmenUndDatenstrukturen}.
\begin{definition}
    Ein Knoten $q \in V$ heißt Quelle, wenn $ \forall p \in V: (p,q) \not \in E
    $, $q$ also nach \autoref{def:kind_eltern} keine Elternknoten hat.
    $s \in V$ heißt Senke, wenn $ \forall p \in V: (s,p) \not \in E $, $s$ also
    nach \autoref{def:kind_eltern} keine Kindknoten hat.
    \label{def:quelle_senke}
\end{definition}

\autoref{def:kind_eltern} definiert die Begriffe Kindknoten, Elternknoten und
Subknoten.
\begin{definition}
    Für zwei Knoten $c$ und $p$ gilt, $c$ ist Kindknoten von $p$, wenn $(p,c)\in E$.
    Umgekehrt gilt, wenn $(p,c)\in E$, $p$ ist Elternknoten von $c$.
    $c$ ist Subknoten von $p$, wenn $\exists P: (v_0=p) \land (v_n=c)$, es also
    einen Pfad von $p$ nach $c$ gibt.
    \label{def:kind_eltern}
\end{definition}
% section Notwendige Grundlagen aus der Graphentheorie (end)

\chapter{Aktueller Stand der Performance Analyse in der \acl{CE}}
\label{sec:analyse}
In der \ac{CE} werden bereits verschiedene Wege genutzt, um die Performance und
das Verhalten der HANA Datenbank zu analysieren. 
\section{Google Benchmark}
\label{sec:google_benchmark}

Google Benchmark ist ein Open Source Benchmarking Tool von Google,
welches es einem ermöglicht einzelne Funktionen in C++ zu benchmarken. 
Dazu kann man ähnlich zu den meisten Test Frameworks drei verschiedene
Codeabschnitte definieren.

Der Hauptabschnitt definiert was genau im Benchmark untersucht werden
soll. Diese legt die für die Messung relevante Logik fest.
Der Setup Abschnitt wird einmal vor jeder Ausführung des Benchmarks
aufgerufen. In dieser werden die Voraussetzungen für die Ausführung des Hauptabschnitts
geschaffen. Man könnte sie beispielsweise nutzen, um Testdaten für den Benchmark
zu generieren oder zu laden, da er nicht bei den Messungen beachtet wird.
Der Teardown Abschnitt wird nach jeder Ausführung des Benchmarks aufgerufen.
Dieser beeinflusst, wie der Setup Abschnitt, die Messung nicht.

Des Weiteren bietet Google Benchmark die Option, einen Benchmark mehrmals mit
unterschiedlichen Parametern durchzuführen. Um beispielsweise die Auswirkung
der Größe des Testdatensatzes auf das Ergebnis zu beobachten.

\autoref{fig:google_benchmark_ausgabe} zeigt eine Beispielhafte Google
Benchmark Ausgabe. Benchmark ist dabei der Name des Benchmarks und hinter dem
Schrägstrich die Größe des Inputs. \foreignlanguage{english}{Time} ist die durchschnittliche Dauer einer
Ausführung des Hauptabschnitts über alle Iterationen hinweg. CPU ist die
durchschnittliche CPU Zeit einer Ausführung des Hauptabschnitts über alle
Iterationen hinweg. \foreignlanguage{english}{Iterations} ist die Anzahl der durchgeführten Wiederholungen
des Hauptabschnitts. Legt man keine Anzahl fest, wird die Anzahl der
Wiederholungen anhand der durchschnittlichen Dauer einer Iteration und der
Varianz der Dauer über alle Iterationen hinweg festgelegt.
\autocite[Vgl.][]{GoogleBenchmark}

\begin{figure}[h]
    \begin{center}
        \begin{verbatim}

---------------------------------------------------------------
Benchmark                          Time         CPU Iterations 
---------------------------------------------------------------
BM_RemoveDetachedNodes/2           4 us        4 us     189823 
BM_RemoveDetachedNodes/8           9 us        9 us      78165 
BM_RemoveDetachedNodes/64         58 us       58 us      12106 
BM_RemoveDetachedNodes/512       469 us      468 us       1515 
BM_RemoveDetachedNodes/4096     4237 us     4234 us        166 
BM_RemoveDetachedNodes/8192    10026 us    10019 us         60 
        \end{verbatim}
    \end{center}
    \caption{Google Benchmark Ausgabe}\label{fig:google_benchmark_ausgabe}
\end{figure}

Google Benchmark wird in der \ac{CE} meistens genutzt, um das Verhalten der
Laufzeit bestimmter Funktionen in künstlich generierten Testfällen zu
vergleichen. Hierbei wird keine HANA Instanz benötigt, da
keine Operationen auf einer Datenbank ausgeführt werden, sondern nur einzelne
Funktionen aufgerufen werden. Es werden folglich
auch keine Testdatensätze für die Datenbank benötigt, sondern nur Daten, auf
welchen man die zu analysierende Funktion aufrufen kann.

Im Vergleich zu anderen Methoden der Performance Messung, die Operationen auf
einer HANA-Instanz ausführen, ist ein Google Benchmark relativ einfach.
Dies liegt daran, dass sie weniger Voraussetzungen erfordern und sich lediglich
auf einen kleinen Teil der Logik konzentrieren.
% section Google Benchmark (end)

\section{HANA Profiler}
\label{sec:hana_profiler}


Eine weitere Methode die Performance und das Verhalten von Software zu
analysieren ist, das bereits in \autoref{sec:arten_performance_analyse}
beschriebene, Profiling.

In der \ac{CE} wird hauptsächlich der \enquote{BOOSS-Profiler}, ein in HANA
integrierter Profiler, verwendet. Auf diesen kann entweder manuell oder im Code
zugegriffen werden. Dabei sind die wichtigsten Befehle \texttt{profiler clear}
um die aktuellen Profilinginformationen zurückzusetzen,  \texttt{profiler
start} um den Profiler zu starten, \texttt{profiler stop} um den Profiler zu
stoppen und \texttt{profiler print} um die gesammelten Profilinginformationen
auszugeben.

Der Profiler kann in diesem Kontext auf zwei verschiedene Arten genutzt werden.
Entweder in dem zur Laufzeit des Profilers Anfragen an eine bestehende
HANA-Instanz gestellt werden. Oder der Profiler wird innerhalb eines Tests oder
Benchmarks aufgerufen und es werden die dort aufgerufen Funktionen gemessen.

Die Ausgabe erzeugt dabei ist dabei zwei gewichtete gerichtete azyklische
Graphen, nach \autoref{sec:grundlagen_graphentheorie}, von denen einer die Verteilung der CPU-Zeit und der andere die
Verteilung der Wartezeit, der aufgerufenen Funktionen beinhaltet. Im Folgenden
werden die beiden Graphen CPU-Graph und Warte-Graph genannt.
\autoref{fig:beispielausgabe_hana_profiler} stellt eine mögliche Ausgabe des
HANA Profilers dar. Die folgende Beschreibung gilt für den CPU- als auch den
Warte-Graphen. Jeder Knoten des Graphen spiegelt eine zur Laufzeit des
Profilers aufgerufene Funktion wider. Jeder Knoten beinhaltet drei
Informationen. Den Namen der aufgerufenen Funktion, sowie den Wert $I$ und den
Wert $E$. $I$ ist der Anteil der Gesamtzeit, welcher von diesem Knoten und all
seinen Subknoten benötigt wurde. $E$ ist der Anteil der Gesamtzeit, welcher
von diesem Knoten benötigt wurde. Folglich gilt für alle Knoten $I\geq E$.
Die Kindknoten eines Knoten $K$ sind die Funktionen, welche von $K$ aufgerufen
wurden.

\begin{figure}[h]
    \begin{center}
        \includegraphics[page=1]{Bilder/pdf/profiler_output_example.pdf}
    \end{center}
    \caption{Beispielausgabe des HANA Profilers}\label{fig:beispielausgabe_hana_profiler}
\end{figure}

% section HANA Profiler (end)

\chapter{Konzept}

Für die Untersuchung des Optimierungsalgorithmus wird ein experimenteller
Ansatz statt einem analytischem gewählt, da die experimentelle Vorgehensweise
es zum einen einfacher macht sehr komplexe Algorithmen zu untersuchen, zum
anderen, eine experimentelle Untersuchung realitätsnähere Ergebnisse liefern
kann \autocite[vgl.][3]{ExperimentalMethods}. Hierzu werden in dieser Arbeit
zwei unabhängige Variablen betrachtet. Zum einen die Größe und zum anderen der
Aufbau des zu optimierenden Modells, diese wurden gewählt, da sie direkt
kontrolliert werden können und erwartet wird, dass sie einen großen Einfluss
auf die Laufzeit haben \autocite[vgl.][506]{ExperimentalAnalysis}. Der Aufbau
wird in dieser Arbeit als Art des Modells bezeichnet. Die gemessene abhängige
Variable ist die Laufzeit des Optimierungsvorgangs. Unabhängige Variablen sind
Variablen, welche aktiv verändert werden, während abhängige Variablen gemessen
werden \autocite[vgl.][236]{EmpirischeMethoden}. Damit Messergebnisse
sinnvoll vergleichen werden können, darf zwischen zwei Messungen nur eine
unabhängige Variable verändern werden. \autocite[vgl.][236]{EmpirischeMethoden}.
Deshalb werden mehrere Messreihen durchgeführt, wobei innerhalb einer Messreihe
der Art konstant ist, die Größe jedoch variabel ist. Für jede Art wird eine
neue Messreihe begonnen. Störvariablen, also Einflussfaktoren, welche ebenfalls
die abhängigen Variablen beeinflussen, jedoch während der Messung
unkontrolliert auftreten, \zB die Prozessorauslastung des Rechners, auf
welchem die Messung durchgeführt wird \autocite[vgl.][237]{EmpirischeMethoden}.
Ist der Prozessor weniger ausgelastet, dann ist die gemessene Zeit vermutlich
geringer, als wenn der Prozessor stark ausgelastet ist.

Um den Einfluss dieser Störvariablen möglichst gering zu halten, wird jede Messung
$n$ mal wiederholt. Aus diesen Messwerten wird nun ein Konfidenzintervall
gebildet, welches mit der Wahrscheinlichkeit $1 - \alpha$ den tatsächlichen
Erwartungswert $\mu$ enthält. Dazu werden die Messwerte als eine T-verteilte
Zufallsvariable $X$ betrachtet, da $n$ aufgrund der Dauer einer Messung nicht
sehr groß gewählt werden kann. Die Varianz $\sigma^2$ von $X$ ist dabei
unbekannt und muss anhand der Stichprobe geschätzt werden, für diese Schätzung
gilt: $\hat{\sigma}^2 = S^2$ \autocite[vgl][528]{Statistik}. Für das
($1-\alpha$)-Konfidenzintervall ergibt sich deshalb nach
\autocite[vgl.][533]{Statistik}:

\begin{equation*}
    [\bar{X} - t_{n-1,1-\alpha/2}\sqrt{S^2/n}, \bar{X} +
    t_{n-1,1-\alpha/2}\sqrt{S^2/n}]
\end{equation*}

Dabei ist $\bar{X}$ der Mittelwert der Stichprobe und $S^2$ die korrigierte
Stichprobenvarianz \autocites[vgl.][59, 502]{Statistik}. 
\begin{equation*}
    \bar{X} = \frac{1}{n} \displaystyle\sum^{n}_{i=1}x_i 
\end{equation*}
\begin{equation*}
    S^2 = \frac{1}{n-1}\displaystyle\sum^{n}_{i=1}(x_i-\bar{X})^2
\end{equation*}

Für die verschiedenen Arten von Modellen, werden Modelle, von reellen Abfragen
betrachtet, um aus diesen, allgemeine Regeln festzulegen, mit welchen man
Modelle variabler Größe aber derselben Art erzeugen kann. Die Modelle werden
für die Messung künstlich erzeugt, um die unabhängigen Variablen gezielt
verändern zu können. Auf die reellen Abfragen wird dabei zurückgegriffen, um
die Relevanz der Messung, für reelle Szenarien zu erhöhen.
\autocite[Vgl.][500f]{ExperimentalAnalysis}

Diese Messdaten werden genutzt, um festzustellen, wie sich die Laufzeit der
Optimierung für Modelle bestimmter Art bei steigender Größe verhalten. Dabei
ist besonders interessant, ob sich die Laufzeit zur Größe linear verhält,
beziehungsweise ob sie stärker oder schwächer ansteigt. Für die Optimierung des
Algorithmus sind nun die Modellarten interessant, bei welchen die Laufzeit
stärker als linear ansteigt, da diese Modellarten ein besonders hohes Potenzial
haben lange Laufzeiten zu verursachen. Um genauer herauszufinden, in welchem
Teil des Optimierungsalgorithmus besonders viel Zeit benötigt wurde, werden
diese Modelle nochmals mithilfe des in \autoref{sec:performance_analyse}
beschriebenen Profilings untersucht.
Dazu werden mehrere Modelle dieser Art
optimiert, während der Profiler protokolliert, in welchen Methoden sich wie
lange aufgehalten wurde. Anschließend muss beurteilt werden, ob die Zeit,
welche in dieser Methode benötigt wird, erwartbar ist oder sie geringer sein
sollte. Beziehungsweise, ob es eine Möglichkeit gibt diesen Teil des
Algorithmus zu beschleunigen.

Wurde eine Optimierungsmöglichkeit gefunden und umgesetzt, kann mit einem
Benchmark, welcher reelle Modelle nutzt validiert werden, ob die Veränderung
eine reale Verbesserung verursacht hat. \todo{Entfernen?}

\chapter{Umsetzung}

\section{Benchmarking}

Da sich die Arbeit auf die Optimierungsfunktion der \ac{CE} beschränkt, kann
sich auch bei den Messungen auf diese beschränkt werden. Um die Leistung einer
bestimmten Funktion zu untersuchen, eignet sich, das in
\autoref{sec:arten_performance_analyse} beschriebenen, Benchmarking. Somit sind
die Messungen präziser auf diese Funktion ausgerichtet und es wird Aufwand
gespart, welcher auftreten würde, wenn man die Untersuchungen anhand einer
HANA-Installation durchführen würde. Genauer wird das in der \ac{CE} verwendete
Benchmarking-Framework Google Benchmarks genutzt. Dieses bietet die Möglichkeit
eine Messung mit mehreren Parametern und mehreren Variationen von diesen
durchzuführen. Es kann \zB für zwei Parameter Größe und Art jeweils eine Menge,
$G$ und $A$, an Werten festgelegt werden, für welche eine
Messung ausgeführt werden soll.

\begin{equation*}
    G=\{2;4;8\}
    \qquad A=\{j;p\}
\end{equation*}


\begin{equation*}
    K = G \times A = \{(2,j);(4,j);(8,j);(2,p);(4,p);(8,p)\}
\end{equation*}

Das kartesische Produkt $K$ ist eine Menge von Tupeln $(g,a)$. Jedes Tupel
spiegelt eine Kombination von Parametern wider, für welche eine Messung
durchgeführt wird \autocite[vgl.][50]{Mengenlehre}. Für alle $(g,a) \in K$ wird
das Modell der Art $a$ und der Größe $g$ optimiert und die Dauer dieser
Optimierung gemessen. Dieses Modell wird im Folgendem als Modell $(g,a)$
bezeichnet.

Die Parameter sind folgendermaßen definiert. Für jede Art wird eine Funktion
definiert, welche ein Modell dieser Art zurückgibt. Die Größe ist ein Wert $n$,
welcher an diese Funktion übergeben wird. Was genau eine Größe von $n$ bedeutet
ist dabei von Art zu Art unterschiedlich. Es kann also sein, dass das Modell
$(4,j)$ mit Größe $4$ und Art $j$ mehr Knoten hat als das Modell $(4,p)$,
obwohl sie dieselbe Größe haben. Das spielt jedoch keine Rolle, da diese
Messungen lediglich dazu dienen, zu untersuchen, ob die Laufzeit linear oder
stärker anwächst. Es ist jedoch wichtig, dass die Anzahl der Knoten, mit der
Größe linear anwächst, da sonst die Kurven nicht mehr vergleichbar währen.
Zwei Modelle sind also von gleicher Art, wenn sie mit der gleichen
Vorgehensweise erzeugt wurden. \autoref{fig:bsp_modell_art} zeigt verschiedene
Modelle. Von diesen könnten \zB \autoref{bsp_modell_art_1} und
\autoref{bsp_modell_art_2} von der gleichen Art sein, da diese erzeugt wurden,
indem ein \foreignlanguage{english}{Aggregation}-Knoten, $n$
\foreignlanguage{english}{Projection}-Knoten und ein
\foreignlanguage{english}{Table}-Knoten hintereinander gehängt werden.
Dabei könnte $n$ die Größe sein und $n+2$ die Anzahl der Knoten.

\begin{figure}
    \begin{subfigure}[b]{0.3\textwidth}
        \begin{tikzpicture}[
    node distance = 1cm,
    request/.style = {rectangle, draw, blue, minimum width=2cm, minimum height=0.8cm},
    op_node/.style = {circle, draw, minimum size=1cm},
    table_node/.style = {circle, draw, magenta, minimum size=1cm},
    arrow/.style = {->, >=stealth}
]

% Nodes
\node[request] (0) at (0,4) {Abfrage};
\node[right=0.2cm of 0] {Aggregation};

\node[op_node] (1) at (0,2) {1};
\node[right=0.2cm of 1] {Projection};

\node[table_node] (2) at (0,0) {2};
\node[right=0.2cm of 2] {Table};

% Arrows
\draw[arrow] (0) -- (1);
\draw[arrow] (1) -- (2);

\end{tikzpicture}
 \caption{Beispiel
        1}\label{bsp_modell_art_1}
    \end{subfigure}
    \begin{subfigure}[b]{0.3\textwidth}
        \input{Bilder/tikz/modell_projektionen_2.tex} \caption{Beispiel
        2}\label{bsp_modell_art_2}
    \end{subfigure}
    \begin{subfigure}[b]{0.3\textwidth} \input{Bilder/tikz/modell_join.tex}
        \caption{Beispiel 3}\label{bsp_modell_art_3}
    \end{subfigure}
    \caption{Beispiel Modelle}\label{fig:bsp_modell_art}
\end{figure}

\begin{listing}
    <@\textcolor[HTML]{724BFF}{\texttt{BENCHMARK}}@><@\textcolor[HTML]{000000}{\texttt{(}}@><@\textcolor[HTML]{000000}{\texttt{BM\_Optimizer}}@><@\textcolor[HTML]{000000}{\texttt{)}}@>
<@\textcolor[HTML]{000000}{\texttt{}}@><@\textcolor[HTML]{000000}{\texttt{\ \ }}@><@\textcolor[HTML]{1041FF}{\texttt{->}}@><@\textcolor[HTML]{724BFF}{\texttt{Apply}}@><@\textcolor[HTML]{000000}{\texttt{(}}@><@\textcolor[HTML]{000000}{\texttt{BenchOptimizerArguments}}@><@\textcolor[HTML]{000000}{\texttt{)}}@>
<@\textcolor[HTML]{000000}{\texttt{}}@><@\textcolor[HTML]{000000}{\texttt{\ \ }}@><@\textcolor[HTML]{1041FF}{\texttt{->}}@><@\textcolor[HTML]{724BFF}{\texttt{Repetitions}}@><@\textcolor[HTML]{000000}{\texttt{(}}@><@\textcolor[HTML]{DE6F10}{\texttt{20}}@><@\textcolor[HTML]{000000}{\texttt{)}}@>
<@\textcolor[HTML]{000000}{\texttt{}}@><@\textcolor[HTML]{000000}{\texttt{\ \ }}@><@\textcolor[HTML]{1041FF}{\texttt{->}}@><@\textcolor[HTML]{724BFF}{\texttt{Unit}}@><@\textcolor[HTML]{000000}{\texttt{(}}@><@\textcolor[HTML]{1F8F42}{\texttt{benchmark}}@><@\textcolor[HTML]{1041FF}{\texttt{::}}@><@\textcolor[HTML]{000000}{\texttt{kMicrosecond}}@><@\textcolor[HTML]{000000}{\texttt{);}}@>
<@\textcolor[HTML]{000000}{\texttt{}}@>
\caption{My Listing}
\label{my_listing}
\end{listing}

\autoref{my_listing}

\todo{Beschreibung der untersuchten Arten (Codeausschnitte anpassen und evntl einfügen)}

\section{Profiling}

\chapter{Umsetzung und Interpretation der Messwerte}
In diesem Kapitel werden die Messwerte aus der Umsetzung dargestellt und
statistisch analysiert. Für die Interpretation der Messwerte werden, wie in
\autoref{tab:zahlen_modell_arten} gezeigt, den verschiedenen Arten von Modellen
Zahlen zugeordnet. 
\begin{table}[h]
\centering
\begin{tabular}{lc}
\toprule
\textbf{Modellart} & \textbf{Zahl} \\
\midrule
PROJECTION & 0\\
JOIN & 1\\
UNION & 2\\
\bottomrule
\end{tabular}
\caption{Zuordnung von Modellart zu Zahl }
\label{tab:zahlen_modell_arten}
\end{table}

\section{Benchmarking Ergebnisse}
\autoref{fig:messungen_all} zeigt die durchschnittlichen Messwerte sowie dem
0,99-Konfidenzintervall für diese Messwerte, für alle Modellarten.
Es lässt sich erkennen, dass für Modellart 0 die Zeit sich annähernd linear zur
Größe verhält. Für die Modellarten 1 und 2 steigt die Steigung mit zunehmender Größe
immer weiter an. Deshalb wird die Optimierung für diese Arten mithilfe von
Profiling genauer betrachtet.

\begin{figure}[h]
    \begin{center}
        \resizebox{!}{8cm}{\includegraphics[page=1]{Bilder/pdf/plot_all_release.pdf}}
    \end{center}
\caption{Durchschnittliche Zeit mit 0,99-Konfidenzintervall}\label{fig:messungen_all}
\end{figure}

\section{Profiling Ergebnisse}
\autoref{fig:profiling_art_1} zeigt einen Ausschnitt aus der Profiler-Ausgabe
für das Modell der Art 1 und der Größe 2048. Dieser Ausschnitt beinhaltet dabei
die relevantesten Knoten der Ausgabe. Der \verb+optimize+-Knoten
hat zwar noch weitere Kindknoten. Aufgrund der geringeren Auswirkung auf die
Laufzeit, wurden diese in der Grafik entfernt, um die Übersichtlichkeit zu
erhöhen. Die relevantesten von diesen Knoten werden jedoch im folgenden weiter
betrachtet. Die Ausgabe für Modellart 2 hat einen ähnlichen Aufbau.

\autoref{tab:e_values} stellt für ausgewählte Knoten den Anteil der von
Funktion dieses Knotens benötigten Zeit von der Gesamtzeit dar. Es sind jeweils
alle Messwerte für die verschiedenen Größen angegeben, um auch das Verhalten
analysieren zu können. Diese Werte entsprechen dem in
\autoref{sec:hana_profiler} erläutertem Wert $E$ aus der Profiler-Ausgabe.
Auffällig ist die \verb+opptimize+-Funktion, diese benötigt sowohl bei
Modellart 1 als auch bei Modellart 2 einen sehr großen Anteil der Zeit. Dieser
Anteil steigt jedoch zumindest bei Modellart 2 nicht eindeutig mit der Größe. Dies
bedeutet jedoch nicht, dass die benötigte Zeit in dieser Funktion ebenfalls
nicht steigt, da diese Angaben prozentual zu der jeweiligen Gesamtdauer sind.
Die anderen Knoten sind deshalb interessant, weil der Anteil an der Gesamtdauer
mit der Größe des Modells zunimmt. Die größte Zunahme von Prozentpunkten ist
dabei $32 \% - 19 \% = 13 \%$, die größte prozentuale Steigerung ist
$\frac{12\%}{1,5\%} - 1 = 700\%$. Aufgrund des hohen Anteils oder der großen
Steigerung von diesem bieten sich alle in \autoref{tab:e_values} dargestellten
Funktion für eine nähere Betrachtung an.

\begin{table}[h]
\centering
\resizebox{\textwidth}{!}{
\begin{tabular}{|l|l|c|c|c|c|}
\toprule
\textbf{Art} & \textbf{Funktion} & \textbf{2048} & \textbf{4096} & \textbf{8192} &\textbf{16384} \\
\midrule
1 & Optimizer2::optimize                           & 18 \% & 22 \% & 24 \% & 24 \% \\ \hline
1 & CombineJoinOverProjectionPattern::doesMatch    & 19 \% & 26 \% & 30 \% & 32 \% \\\hline
1 & BuildExpresionFilterPattern::doesMatchInternal & 1,5 \% & 6,1 \% & 11 \% & 12 \% \\\hline
2 & Optimizer2::optimize                           & 37 \% & 32 \% & 33 \% & 34 \% \\ \hline
2 & BuildExpresionFilterPattern::doesMatchInternal & 4,2 \% & 10 \% & 10 \% & 14 \% \\
\bottomrule
\end{tabular}
}
\caption{Anteil an der Gesamtlaufzeit ausgewählter Knoten}\label{tab:e_values}
\end{table}

\include{Inhalt/fazit}

% ---- Anhang
\appendix
\chapter{Anhang}
Hier ist der Anhang
%\clearpage
%\pagenumbering{Roman}  % römische Seitenzahlen für Anhang

% ---- Literaturverzeichnis
\cleardoublepage
\renewcommand*{\chapterpagestyle}{plain}
\pagestyle{plain}
%\pagenumbering{Roman}                   % Römische Seitenzahlen
%\setcounter{page}{\numexpr\value{savepage}+1}
\printbibliography[title=Literaturverzeichnis]


\newpage
\end{document}
